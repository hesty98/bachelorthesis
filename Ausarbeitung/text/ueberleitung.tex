\subsection{Eingrenzung des Themas}
Nach der Präsentation der Ergebnisse des Forschungsseminars am 7. Februar 2020 in Braunschweig wurde in Absprache mit den Betreuern der Umfang der Bachelorarbeit eingegrenzt. Die Aspekte des User-Centered-Designs werden im Kontext der Arbeit nicht weiter beachtet, statt dessen soll ein stärkerer Fokus auf die Bausteine der Wertschöpfungskette und den zu entwickelnden Prototypen gelegt werden, um die Bedarfserkennung und Bereitstellung von Software möglichst detailliert darstellen zu können.\\
Des weiteren wurde sich darauf geeinigt, den im Prototypen dargestellten Anwendungsfall zu ändern. Zuvor war vorgesehen, einem Fahrzeug anhand der im Forschungsseminar skizzierten Suchalgorithmen eine neue Software vorzuschlagen. In dem neuen Anwendungsfall wird ein Fahrzeug eine Software installieren und nutzen können, welche es dem Fahrzeug ermöglicht selbstständig auf einem Parkplatz parken zu können.Die Ergebnisse des Forschungsseminars, insbesondere die skizzierten Suchalgorithmen und die Integration von OpenScenario-Dateien werden daher \textbf{nicht} im Prototypen implementiert, finden aber dennoch in der Bachelorarbeit Einfluss. Durch die Änderung des Anwendungsfalls konnten weitere Teilnehmer der Supply Chain identifiziert und somit ein umfassender Überblick des Marktes für die Erkennung und Bereitstellung neuer Fahrumfänge erstellt werden.