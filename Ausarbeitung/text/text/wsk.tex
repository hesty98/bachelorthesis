\section{Der Markt neuer Fahrumfänge}\label{wsk}
Damit die Verteilung von \textbf{Software} vernünftig organisiert wird, bedarf es einer Plattform über welche Softwares angeschafft und heruntergeladen werden können. Diese zentrale Verwaltung von Softwares wird in Form eines Softwareshops realisiert. Über diesen können Softwares von Fahrzeughaltern orts-und zeitunabhängig gekauft und installiert werden. Der Shop stellt den Mittelsmann zwischen Fahrzeughaltern und den Softwareherstellern dar. Die installierten Softwares können unter anderem den Fahrumfang autonomer Fahrfunktionen erweitern und das Auto mit anderen Akteuren des Straßenverkehrs verbinden. Der Fahrzeughalter kann mit diesen Interagieren und von Ihnen bereitgestellte \textbf{Services} \textit{(z.B. Kauf eines Parktickets)} nutzen.\\
Die Bereitstellung eines Automarkenübergreifenden Softwareshops ist aus mehreren Gründen sinnvoll:\\

\textbf{1. Nachhaltigkeit von Fahrzeugen}\\
Nach dem verlassen des Fließbandes altert die Software eines Autos. Updates sind heutzutage nur beim Mechaniker möglich und ist zudem mit einem großen Aufwand verbunden. Durch eine Kabellose Schnittstelle sollen Fahrzeughalter gewünschte Software orts- und zeitunabhängig installieren können. Durch Softwares können Fahrzeuge sicherer\textit{(weniger Unfälle)} und schonender\textit{(geringere Getriebeabnutzung etc.)} gefahren werden, wodurch sich die Lebenszeit des Fahrzeugs um eine unbestimmte Zeitspanne verlängern und langfristig die Nachfrage an Neuwagen zurückgehen kann.\\


\textbf{2. Ressourcen von Fahrzeugen schonen}\\
Durch die stetig steigende Anzahl an installierten Softwares eines Fahrzeugs, werden Festplatten immer voller, die benötigte Rechenzeit kann steigen und der Fahrzeughalter hat keinen Überblick mehr über die installierten Softwares. Durch einen Softwareshop wird es möglich, dass ein Fahrzeug nur tatsächlich benötigte Software installiert, wodurch die Ressourcen geschont werden.\\

\textbf{3. Kosten für Kunden Skalierbar halten}\\
Müsste ein Fahrzeughalter jede neu entwickelte Software auf seinem Fahrzeug installieren und zusätzlich noch für diese Zahlen, können für diesen unnötige Kosten entstehen. So brauch ein Fahrzeug das nur in der Stadt fährt beispielsweise keine Software die das Off-Road fahren unterstützt. Durch die Möglichkeit einzelne Softwares orts- und zeitunabhängig zum Fahrzeug hinzufügen zu können, wird dieses Problem eliminiert.\\

\textbf{4. Verbesserung der Wettbewerbssituation}\\
Durch die Bereitstellung eines Automarkenübergreifenden Softwareshops kann ein großer Teil des Marktes gewonnen werden. Dies kann die Position des Unternehmens in diesem manifestieren und so zu neuen Einnahmequellen führen.\\\\

Um einen Überblick des Marktes zu geben, wird zunächst ein Business Model Cancvas \textit{(BMS)} erarbeitet. Anschließend werden relevante Bausteine der Wertschöpfungskette anhand der Erkenntnisse aus dem Forschungsseminar sowie des Business Models identifiziert und deren Aufgaben erläutert. Abschließend erfolgt die theoretische Ausarbeitung eines Software-Absatzprozesses, welcher im Rahmen des Prototypen \textit{(Kapitel \ref{prototyp})} implementiert wird. 

\subsection{Business Model Canvas}
Das 2004 von Alexander Osterwalder entwickelte Busines Model Canvas \textit{(BMC)} schafft einen Überblick eines Unternehmens und hilft, sich auf die wesentlichen Zielsetzungen zu fokussieren.\footnote{https://ut11.net/de/blog/dein-geschaftsmodell-kompakt-der-business-model-canvas/} Folgend werden die Kundensegmente, die Kundenbeziehungen, die Marketingkanäle und die Einnahmequellen betrachtet. Unter Beachtung dieser werden anschließend die Nutzenversprechen sowie die Schlüsselressourcen und -aktivitäten bestimmt. Durch die abschließende Bestimmung von Schlüsselpartnern und der möglichen Kostenstruktur eines Softwareshops wurden "alle wesentlichen Elemente eines Geschäftsmodells in ein skalierbares System gebracht."\footnote{https://www.startplatz.de/startup-wiki/business-model-canvas/}

\subsubsection{Kundensegmente}
Die Kundensegmente des Marktes lassen sich in Einzelkunden, Gruppen und Flottenbetreiber aufteilen. Einzelkunden sind der übliche Autofahrer, der ein privates Fahrzeug besitzt und Software auf dieses Installieren möchte. Gruppen sind Einzelkunden, die gemeinsam Software kaufen. Hierdurch sollen Familienkäufe aber auch Käufe mit Freunden ermöglicht werden. Beiden Segmenten bedarf es einer gelungenen, übersichtlichen und intuitiven Verwaltung von Software sowie der Unterstützung beim Kauf neuer Software. Flottenbetreiber lassen sich in zwei weitere Segmente aufteilen: \textit{"Leihe und Leasing"} umfasst Autovermieter, Unternehmen die ihren Mitarbeitern Leasingwagen bereitstellen, aber auch Car-Sharing Unternehmen und weitere. Deren Kunden erwarten eine grundlegende Sammlung an Software im Mietfahrzeug vorzufinden und wollen selbstständig weitere Softwares auf eigene Rechnung installieren können. Neben Leihe und Leasing sind auch Unternehmen mit Firmenwagen ein gesondertes Kundensegment. Diese haben kleine oder große Flotten an Fahrzeugen und müssen Softwarekäufe dementsprechend skalieren können. Sie könnten Mengenrabatte bei entsprechend großen Käufen erwarten.\\
Die Wichtigkeit des Einzelkundensegments sticht bei näherem betrachten deutlich hervor, da sie auch in der Absatzkette der anderen Kundensegmente \textit{(Flottenbetreiber in Form von Fahrzeugmietern o.Ä.)} in Erscheinung treten. Dadurch stellen sie einen starken Einflussfaktor auf den Erfolg des Softwareshops dar. Aufgrund dieser Wichtigkeit ist eine nähere Unterteilung des Segments sinnvoll:\\
\begin{itemize}
	\item[\textbf{18-25}] 
	Die Generation Z und deren ältere Geschwister sind mit modernen Technologien wie dem Computer und dem Smartphone aufgewachsen. Für sie ist es intuitiv zu Alltagsunterstützenden Hilfestellungen zu greifen. Zugleich stellt dies jedoch die Gruppe mit dem geringsten Einkommen dar. Deshalb sollte sie nicht nur zum Kauf von Software bewegt werden, sondern auch für die Vermarktung in den älteren Kundensegmenten genutzt werden.
	\item[\textbf{25-33}]
	Ebenfalls bestens mit Technik vertraut, umfasst dieses Segment die vermutlich wichtigsten Kunden. Die Präsenz und Vernetzung im Internet dieser Gruppe ist ähnlich zu dem vorherigen Segment sehr hoch. Personen dieser Gruppe sind in der Regel dabei, sich ein geordnetes Leben aufzubauen in Form einer Karriere, einer Familie oder anderem. Sie sind finanziell besser aufgestellt als die jüngeren Segmente, können also quantitativ mehr Software beziehen. Sicherheit, Datenschutz und Privatsphäre spielen für dieses Segment eine wichtige Rolle, da sie die schlechten Seiten einer Vernetzten Welt kennen und hierdurch vorsichtig agieren. Durch ihr fortgeschrittenes Alter sind sie in den Augen älterer Generation Vertrauenswürdig und möglicherweise der Ansprechpartner im Bezug auf Technik.
	\item[\textbf{33-50}]
	Fest im Leben stehend, stellt dieses Segment das Mengenmäßig größte dar. Im Gegenteil zu den jüngeren Segmenten ist hier wahrscheinlich dass die meisten ein eigenes Auto haben\textit{(was in den vorherigen Gruppen nicht immer der Fall ist)}. Das Auto ist hier ein Statussymbol und ein Alltagsgegenstand für die Vielzahl an Kurzstrecken die zurückgelegt werden. 
	\item[\textbf{50-65}]
	Auch in diesem Segment haben die meisten ein eigenes Auto. Dieses wird sich öfters geteilt, da die Notwendigkeit für Zwei Autos nicht mehr gegeben ist oder die mittlerweile erwachsenen Kinder das Auto der Eltern nutzen. Die Anforderungen sind vergleichbar zu denen der 33-50 Jährigen, jedoch lässt sich diese Gruppe im Bezug auf neue Technik eher beraten.
	\item[\textbf{65-75}]
	je älter der Kunde ist, desto geringer ist die durchschnittliche Technikaffinität. In diesem Segment sind die Mehrwerte von Technik maßgebend dafür, ob ein Kauf stattfindet oder nicht. Eröffnet sich Raum zur Kritik, neigen diese eher vom Kauf ab. Zeitgleich sind sie von den jüngeren Segmenten einfach zu beeinflussen wenn es um den Kauf neuer Technik geht. 
	\item[\textbf{75 +}]
	Durch das fortgeschrittene Alter stellt dieses Segment oftmals eine große Gefahr für den Straßenverkehr dar. Es legt Wert auf ein weitgehend selbstständig fahrendes Fahrzeug, da so Strecken zurückgelegt werden können die im Normalfall zu unsicher zurückzulegen wären. 
\end{itemize}
\subsubsection{Kundenbeziehungen}
Über die Wichtigkeit des Einzelkundensegments ist eine einheitliche Behandlung von Kundensegmenten wichtig. Damit Fahrzeughalter nach einem Erstkauf auch weitere Transaktionen in Betracht ziehen, müssen deutlich Erkennbare Mehrwerte geliefert werden. Unterstützend hierfür sind die Grundbausteine der Kundenbeziehung \textit{Vertrauen aufzubauen, Unterstützung bei Kaufentscheidung bereitstellen, Individuelle und zuvorkommender Kundenservice}. Die Akquise von Software soll \textit{fließend} sein, um nicht als Akt hohen Aufwands wahrgenommen zu werden.

Der Aufbau einer guten Kundenbeziehungen kann gut durch gratis Software oder Software-Gutscheine initialisiert werden. Die ersten Akquisitionen sollten einen deutlichen Mehrwert für den Käufer bieten, um eine möglichst hohe Zufriedenheit bei diesem zu erzielen. Damit eine positive Kundenbeziehung bestehen bleibt, wird die Verwaltung und Überwachung von Software teilweise durch die Systeme übernommen. Werden Softwares nicht genutzt oder es ergeben sich Möglichkeiten für den Fahrzeughalter Geld zu sparen, weißt der Softwareshop hierauf hin. Eine langfristige Kundenbindung kann durch die Einbindung dessen in das Brainstorming für neue Softwares zu integrieren. Kunden können so ihre Wünsche erfüllt bekommen und es werden ihre größten Wünsche im Bezug auf ihr eigenes Fahrzeug können in Erfüllung gehen.\\
Da Flottenbetreiber mit einem Softwarekauf größere Summen ausgeben und somit auch ein größeres Finanzielles Risiko eingehen, bedarf es einen höheren Grad der individuelle, nicht automatischem, Beratung. Hierdurch können Fehler präventiert werden und somit die Zufriedenheit des Segments sichern.
\subsubsection{Marketingkanäle}
Um weitblickend die marktführende Plattform für die Erkennung und Bereitstellung von Fahrfunktionssoftware zu sein, sollte ein Markteintritt auf allen möglichen Marketingkanälen stattfinden. Die Altersgruppen des Einzelkundensegments werden über unterschiedliche Kanäle erreicht. Um ältere Segmente (50+) zu erreichen, ist das Nutzen \textit{traditioneller} Medien wie Plakate, Zeitschrift oder dem Fernsehen Sinnvoll. Zu Sendezeiten, in denen überwiegend junge Zuschauer aktiv sind, sollen die Werbespots dementsprechend abgeändert werden. Jüngere Kunden sind aktiver auf anderen Medien oder Plattformen, wie Instagramm, Facebook, Twitter, aber auch und vor allem Youtube. Über diese kann zum einen Influencermarketing \textit{(Influencer bewerben die Plattform)} betrieben werden oder auch klassische Werbeanzeigen geschaltet werden. Im Fokus stehen Menschen, die permanent unter Zeitdruck sind, da sie durch ein autonom Fahrendes Fahrzeug viel zeit einsparen können.

\subsubsection{Nutzenversprechen}
\textbf{Orts- und Zeitunabhängige Akquise eigens ausgewählter Fahrfunktionen}\\
Statt wie in der Vergangenheit Softwareupdates für das Fahrzeug nur beim Mechaniker installieren zu können, können Fahrzeughalter über den Softwareshop immer und überall Software akquirieren. Die installierten Funktionen können selbstständig gewählt werden, wodurch Fahrzeuge individuell gehalten werden können.\\\\
\textbf{Überwachung, Verwaltung und Vorschlagen von Software}\\
Fahrzeughalter sollen überwachen können, welche Software wie oft genutzt wird nicht benötigte Softwares einfach deinstallieren zu können. Neben diese Unterstützung bei der Entdeckung nicht Benutzer Software, wird auch die Suche nach einem möglichen Softwarebedarf für Fahrzeughalter automatisiert. Dies kann vor allem weniger technikaffine Kundensegmente unterstützen und Sicherheit im Umgang mit dem Fahrzeug schaffen. Wie der Bedarf einer Software erkannt werden kann, wird in Kapitel \ref{2.3} erläutert.\\\\
\textbf{Autofahren wird sicherer}\\
Software fährt sichere als Mensch, da keine äußeren Einflüsse wie Stress, Müdigkeit, Wut oder anderes die Fahrsicherheit negativ beeinflusst. Neben den Sicherheitsvorteilen einzelner Fahrsysteme eines Fahrzeugs ist weitblickend vor allem die Sicherheit, welche durch C2X-Kommunikation ermöglicht wird der entscheidende Faktor der die Sicherheit weiter erhöhen wird.\\\\
\textbf{Lebenszeit des Autos wird verlängert}\\
Aufgrund der gestiegenen Sicherheit wird ein Fahrzeug künftig unfallfreier Fahren. Des weiteren wird die Fahraufgabe von einer Software schonender für Getriebe- und Motorteile durchgeführt. Beides ermöglicht eine durchschnittlich längere Lebenszeit für ein Fahrzeug, was auch die Anschaffung eines neuen Fahrzeugs verzögern kann.\\\\
\textbf{Geringerer Wertverlust von Fahrzeugen}\\
Durch das stetige Nachladen von Fahrfunktionen kann ein Fahrzeug auf dem aktuellen Stand der Technik bleiben. Hierdurch werden Unterschiede bezüglich der Fahrfunktionen von Fahrzeugen überschaubar und der Wertverlust eines Fahrzeugs ist dementsprechend geringer.\\\\
\textbf{Fahrzeughalter gewinnen Zeit}\\
Durch das Nutzen autonomer Fahrfunktionen können immer weitere Strecken zurückgelegt werden ohne die Steuerung zu übernehmen. Die hierdurch gewonnene Zeit kann zum Arbeiten, oder auch zum verbringen von Zeit mit den anderen Insassen des Fahrzeugs verwendet werden. Diese Zeit war vorher durch die Fahraufgabe belegt.\\\\
\textbf{Stetige Erweiterung des Softwareangebots}\\
Durch die Bereitstellung eines offenen Marktes für Softwareanbieter, wird der entstehende Wettbewerb die Entwicklung neuer Softwares mit unterschiedlichen Funktionen antreiben. Das Spektrum an möglichen Softwares wird zunehmend größer werden und mehr "Probleme" des Alltags bewältigen.\\\\
\textbf{Interaktion mit der Autoumwelt}\\
Bestimmte Softwares ermöglichen die Kommunikation mit anderen Akteuren der Umwelt \textit{(Ampeln, Parkplätze, andere Fahrzeuge, Drive-In-Restaurants, uvm.)} und schaffen so neue Möglichkeiten miteinander zur Interaktion mit diesen über das Auto.

\subsubsection{Einnahmequellen}
Der Softwareshop hat Zwei unterschiedliche Einkommensströme. Zum einen kann es eine 'Anmeldegebühr' geben, die Softwareprovider zahlen müssen um Software im Shop veröffentlichen zu können. Außerdem können diese ihre Softwares im Shop bewerben, indem sie Werbeflächen kaufen auf denen ihre Software zum Kauf angeboten wird. Dies ist Vergleichbar mit dem Ergebnis einer google-Suche, in welcher Ganz oben \textit{gepsonserte} Internetseiten/Produkte zu finden sind.\\
Neben Einnahmen durch die Softwareprovider wird auch Umsatz durch den kauf und die Nutzung von Software generiert. Beim Kauf einer Software geht ein fester Prozentsatz des Verkaufspreises an den Betreiber des Shops. Zum Vergleich: Bei Android Apps liegt der Anteil den Google einnimmt bei 30\%. Auch bei entstehende In-App-Käufen \textit{(z.b. für die Nutzung eines \textbf{Services})} geht ein Anteil von 10\% an den Shop. Die vorgestellten Kundensegmente interagieren dementsprechend nur mit Software- und Serviceprovidern.\\
Durch die Integration von Entwicklungs- und Analyse-Tools können weitere Einnahmen generiert werden. Provider können diese nutzen und werden durch sie bei der Entwicklung neuer Softwares unterstützt.

\subsubsection{Schlüsselressourcen}


Benötigte Ressourcen zum erfüllen der Nutzenversprechen
Rssourcen für Distribution, Kundenbeziehungen, Erlösquellen
Software\\
Nutzerdaten (?)\\

\subsubsection{Schlüsselaktivitäten}
Welche Aktivitäten erfüllen Kundennutzen?
Welche Aktivitäten notwendig Vertriebskanäle und Kundenbeziehungen.

implementierung\\
Ideenfindung\\
Kundenservice\\
Weiterbildung der Fahrzeughalter (Marketing)\\


\subsubsection{Schlüsselpartner}

\textbf{SoftwareProvider}\\
Die wichtigsten Partner stellen Softwareprovider dar. Diese befüllen die DriveAround-Plattform mit entwickelten Applikation, welche entsprechend der Sicherheitsrichtlinien\ref{sicherheitsrichtlinien} von DriveAround entwickelt wurden. Diese Art Partnerschaft ist auch im Androidmarkt zu sehen, wo App-Entwickler den Play Store befüllen. Darüber hinaus schafft diese Entwickler-Community auch neue Ideen für das Android-Framework, ein Aspekt der auch in diesem Markt greifen kann.\\\\
\textbf{ServiceProvider}\\
ServiceProvider sind Akteure der realen Welt, welche über eine Software mit den Fahrzeuginsassen interagieren möchte. Sie bieten im Gegenteil zu Softwareprovider Güter und Dienstleistungen an, die aktiv bemerkbar sind. Dies könnten Parkplätze, Drive-In-Restaurants ohne Wartezeit oder anderes sein. Serviceprovider arbeiten mit Softwareprovidern zusammen. So soll verhindert werden, dass jeder Parkplatz einer Stadt eine eigene Applikation benötigt. Ein Softwareprovider stellt also, ähnlich wie Lieferando\footnote{quelle}, eine Plattform zur Verfügung bei der sich andere Unternehmen anmelden können.

\textbf{Netzwerkbetreiber}\\

\subsubsection{Kostenstruktur}
Tabelle machen mit Kundengruppen in Spalten und die ganzen Attribute eines ANgebots(Forschungsseminar) in Zeilen\\

\subsection{Kernbestandteile der Wertschöpfungskette}\label{anwendungsfall}

\textbf{Hier ist geplant:}
\begin{itemize}
	\item Neue Akteure der Wertschöpfungskette erläutern\\
	Wie machen sie Geld? Welche Mehrwerte entstehen durch die neuen AKteure? Wie groß wird dieser Markt?
\end{itemize}

Grundlagen des Markts:\\
Softwarearten unterscheiden die gekauft werden können\\
Beteiligte Stakeholder der WSK\\
OEM \\
Datenanalyse, Bedarfsentdeckung und ServerSchnittstelle\\
Softwareprovider\\
Softwareentwicklung, Bedarfsentdeckung \\
ServiceProvider\\
Vollstrecker der eigentlichen DIenste (bei Services)\\
Konzept einer (modernen) Wertschöpfungskette\textit{ (eher eine SupplyChain)}\\
Primäraktivitäten sind der Erkennung bzw. der Bereitstellung zuzuordnen\\\\
\textbf{Erkennung}: \\
	Einkauf \& Lagerung\\
	Produktion\\\\
\textbf{Bereitstellung}:\\
	Marketing\\
	Lieferung\\
	Service\\

\textbf{Unterstützende Aktivitäten (Entlang der Erkennung Bereitstellung):}\\
	Unternehmensinfrastruktur\\
	Personalwirtschaft\\
	technologieentwicklung\\
	Beschaffung\\

\subsection{Absatzprozess im Kontext von automatisierten Parken}\label{absatzprozess}
%\textbf{Teilnehmer:\\}
%SW Anbieter\\
%Serviceanbieter\\
%OEM\\
%Fahrer\\

%\textbf{Finanziell:\\}
%. ServiceAnbieter könnten Abgaben an die OEMs(Oder Stadt, Landkreis, oÄ) leisten müssen. Neue Einnahmequelle!\\
%. Fahrer entscheiden lassen, ob er Service nutzen will -> Dies ist wichtig um einen 'Kaufvertrag' abzuschließen (eContracting) 
\subsection{Zusammenfassung}