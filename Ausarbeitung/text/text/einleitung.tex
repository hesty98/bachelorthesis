\setcounter{page}{1}
\section{Motivation}
Das autonome Fahren ist die Zukunft des Automobils. Wann Fahrzeuge jedoch tatsächlich vollständig fahrerlos Fahren ist noch nicht absehbar. Künftig entwickelte Softwares können Fahrzeugen einzelne autonome Fahrfunktionen hinzufügen, durch welche diese immer selbstständiger fahren können. Damit Softwares auch nach dem Kauf eines Fahrzeugs vom Fahrzeughalter installiert werden können, soll der Kauf und die Installation dieser über eine kabellose Schnittstelle möglich sein. Dies kann durch die Integration eines Software-Shops in das Fahrzeug realisiert werden über welchen Fahrzeughalter eigenständig auswählen können welche Softwares auf diesem installiert werden sollen. Um dies verwirklichen zu können muss festgestellt werden, \textit{was relevante Bausteine einer Wertschöpfungskette für die Erkennung und Bereitstellung neuer Fahrumfänge für intelligente Fahrzeuge sein können und wie Automobilhersteller diese umsetzen könnten.} Die Ergebnisse der Forschungsfrage sollen in einem Prototypen veranschaulicht werden, welcher den Kauf, die Installation und Nutzung einer Software darstellt.\\\\ 
Zunächst werden in Kapitel \ref{forschungsseminar} Grundlagen, bestehende Technologien und eigens erstellte System-Konzepte vorgestellt, welche der Bedarfserkennung und Bereitstellung von Software zuzuordnen sind und diese unterstützen. In Kapitel \ref{markt} wird anhand eines Business Model Canvas ein Überblick des Marktes für Fahrfunktionssoftware geschaffen und die in diesem identifizierten Schlüsselaktivitäten werden daraufhin in einer Wertschöpfungskette geordnet und erläutert. In Kapitel \ref{technische_konzepte} wird ein Konzept zur Klassifizierung von Software vorgestellt, nach der Art der unterschiedlichen Zugriffsrechte auf die Systeme des Fahrzeugs. Für die individuelle Bedarfsbestimmung von Software wird ein Suchalgorithmus skizziert und es werden Kommunikationsprotokolle erstellt, welche die Interaktion zwischen dem Server, einem Fahrzeug und anderen Akteuren der Umwelt darstellt. In Kapitel \ref{prototyp} wird schließlich der Prototyp vorgestellt, welcher die Ergebnisse der vorigen Kapitel visuell veranschaulichen soll.\\