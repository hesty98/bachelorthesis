\setcounter{page}{1}
\section{Motivation}
Autonome Fahrzeuge sind die Zukunft des Automobils. Bevor Fahrzeuge jedoch vollständig autonom Fahren, werden 








Wird ein Blick in die nahe Zukunft der Automobilindustrie geworfen eröffnet sich ein  breites Feld neuartiger Technologien. Autonomes fahren, elektro-betriebene Motoren, C2X-Kommunikation, Car-Sharing und weiteres versprechen "eine Zukunft, in welcher sich unser Verständnis des Automobils grundlegend ändern wird."\footnote{Herbert Dies} Durch die neuen Technologien kommt es zu Verschiebungen in den Absatzkanälen von Automobilherstellern. Der klassische Verkauf von Neuwagen wird zurückgehen und durch neu entstehende Seitenmärkte, wie beispielsweise dem Erkennen und Bereitstellen neuer Fahrumfänge verdrängt.\footnote{ericcson}\\

Aktuell sind Fahrzeuge bereits dazu in der Lage, gewisse Verkehrssituationen wie Einparken oder das Fahren im Stau selbstständig zu bewältigen. In absehbarer Zeit wird sich das Spektrum dieser Applikationen vergrößern und Fahrzeughalter sollen dazu in der Lage sein, diese neuen Applikationen auf ihrem Fahrzeug zu installieren, um somit dessen Fahrumfang stetig zu erweitern. Fahrzeughalter sollen bei der Entscheidung unterstützt werden, welche Applikationen sie tatsächlich benötigen, damit Kosten als auch die Ressourcen des Fahrzeugs zu sparen.\\

Um auf diese Änderungen vorzubereiten, wird im Rahmen dieser Arbeit ein Konzept einer Wertschöpfungskette aus der Sicht eines Automobilherstellers erarbeitet. Darüber hinaus wird eine Simulation entwickelt, welche im Kontext automatischer Parkplatzfindung einen Absatzprozess von Software darstellt. Die Entwicklung dessen orientiert sich an dem UPTANE-Standard\footnote{uptane}, welcher im Hinblick auf eine sichere Kommunikation zwischen Fahrzeugen und Servern entwickelt wurde.\\