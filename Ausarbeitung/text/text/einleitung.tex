\setcounter{page}{1}
\section{Motivation}
Das autonome Fahren wird als die Zukunft des Automobils gesehen. Wann Fahrzeuge jedoch tatsächlich vollständig fahrerlos Fahren ist noch nicht absehbar. Künftig entwickelte Softwares können Fahrzeugen einzelne autonome Fahrfunktionen hinzufügen, durch welche diese zunehmend mehr selbstständig fahren können. Damit diese Softwares auch nach dem Kauf des Fahrzeuges selbstständig installiert werden können, soll der Kauf und die Installation dieser über eine kabellose Schnittstelle ermöglicht werden. Dies kann durch die Integration einer Shop-Plattform in das Fahrzeug realisiert werden, über welche Fahrzeughalter eigenständig auswählen können welche Softwares auf dem Fahrzeug installiert werden sollen. Um dies verwirklichen zu können muss festgestellt werden, \textit{was relevante Bausteine einer Wertschöpfungskette für die Erkennung und Bereitstellung neuer Fahrumfänge für intelligente Fahrzeuge sein können und wie Automobilherstellern diese umsetzen könnten.} Es wird ein Prototyp erstellt, welcher den Kauf, die Installation und Nutzung einer Software darstellt.\\\\ 
In Kapitel \ref{markt} wird anhand eines Business Model Canvas ein Überblick des Marktes für Fahrunktionssoftware geschaffen. Die dort identifizierten Schlüsselaktivitäten werden daraufhin erläutert und in eine Wertschöpfungskette eingearbeitet. In Kapitel \ref{technische_konzepte} werden technischen Konzepte vorgestellt, die einzelne Bausteine der Wertschöpfungskette zusätzlich detaillieren. Es wird ein Konzept zur Klassifizierung von Software vorgestellt, nach der Art der unterschiedlichen Zugriffsrechte auf die Systeme des Fahrzeugs. Für die individuelle Bedarfsbestimmung von Software wird ein Suchalgorithmus skizziert und es werden Kommunikationsprotokolle erstellt, welche die Interaktion zwischen dem Server, einem Fahrzeug und Serviceprovidern darstellen. In Kapitel \ref{prototyp} wird der Prototyp vorgestellt, welcher die Ergebnisse der vorigen Kapitel zusammenfassen und visuell veranschaulichen soll. Hierzu wird zunächst die Architektur des Prototypen vorgestellt und verdeutlicht, an welchen Stellen die jeweiligen Konzepte integriert wurden.\\
Zunächst werden hierzu bestehende Technologien und eigens erstellte System-Konzepte vorgestellt, welche die Bedarfserkennung und Bereitstellung von Software unterstützen. 