\subsection{Zusammenfassung}
Das in Kapitel \ref{bmc} erstellte Business Model Canvas bietet einen guten Überblick der Aufgaben eines Software Shops und der damit einhergehenden Kosten, Partner und aufzubauenden Strukturen. Es stellt dar, was Unternehmen bei der Erstellung eines Software-Shops beachten müssen wie gewisse Mehrwerte für Fahrzeughalter geschaffen werden können. Die Auflistung von Nutzenverprechen, Schlüsselpartnern und sonstigen Grundlagen des Shops wäre durch eine Wertschöpfungskette alleine nicht möglich gewesen. Der geschaffene Überblick des Shops hat die anschließende Erstellung der Wertschöpfungskette erleichtert, da einzelne Aspekte der Bedarfserkennung und Bereitstellung von Software ohne weitere Erklärungen eingebaut werden konnten.\\\\
Durch die Einordnung der identifizierten Schlüsselaktivitäten bzw. Bausteine in eine Wertschöpfungskette wurde veranschaulicht, wie diese miteinander zu verknüpfen sind um eine Wertschöpfung für den Kunden zu erreichen. Es wurden Abhängigkeiten und Rollen einzelner deutlich, wie zum Beispiel die Rolle der Sicherheitsverifikation als \glqq\textit{Gatekeeper}\grqq des Shops. Durch die Integration von Bausteinen des \textit{generellen Betriebs}, des \textit{Supply Chain Managements} und sonstigen, welche alle nur indirekt mit der Bedarfserkennung und Bereitstellung von Software zu tun haben, ist die Wertschöpfungskette vollständiger und realistischer geworden.\\\\
Um im Prototypen den Weg von Bedarfserkennung über Bereitstellung hin zur eigentlichen Nutzung von Software abbilden zu können werden im weiteren Verlauf technische Konzepte vorgestellt.