\subsection{Zusammenfassung}
Das in Kapitel \ref{bmc} erstellte Geschäftsmodell bietet einen guten Überblick das Software Shops und die damit einhergehenden Aufgaben, Kosten, Partner und Strukturen der Unternehmung. Anhand der bestimmten Kundensegmente konnten Kundenbeziehungen definiert und Aspekte des Marketings identifiziert werden. Weiterhin wurden Nutzenversprechen aufgestellt, anhand welcher die Schlüsselressourcen, -aktivitäten und -partner identifiziert wurden.\\

In Kapitel \ref{wsk} wurden die zuvor im Geschäftsmodell identifizierten Schlüsselaktivitäten sortiert und den einzelnen Aktivitäten einer Wertschöpfungskette nach Porter zugeordnet. Durch die weitere Aufteilung der Bausteine in fünf Gruppen \textit{(Bedarfserkennung, Bereitstellung, Kundenzufriedenheit, Supply Chain Management, Sonstiges)} konnte verdeutlicht werden, welchen Sinn einzelne Bausteinen haben steckt und worauf diese Einfluss nehmen.\\

Um den Ablauf eines Softwareeinkaufs im Prototypen darzustellen, werden im folgenden technische Konzepte vorgestellt die in diesem implementiert werden können.