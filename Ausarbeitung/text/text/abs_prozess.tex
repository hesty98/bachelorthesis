%\subsection{Absatzprozess im Kontext von automatisierten Parken}\label{absatzprozess}
%1. Auswahl von Software oder Vorschlag für Software ansehen\\
%2. Kaufoption wählen und zahlen\\
%3. Software herunterladen\\
%4. Software nutzen und warten\\
%4.1 Services nutzen\\
%
%
%Schritte des Kunden\\
%1. Schritt erklären\\
%2. geschaffener Wert\\
%2.1 für kunden\\
%2.2 für Unternehmen\\
%
%
%Finanzprognosen
%%\textbf{Teilnehmer:\\}
%%SW Anbieter\\
%%Serviceanbieter\\
%%OEM\\
%%Fahrer\\
%
%%\textbf{Finanziell:\\}
%%. ServiceAnbieter könnten Abgaben an die OEMs(Oder Stadt, Landkreis, oÄ) leisten müssen. Neue Einnahmequelle!\\
%%. Fahrer entscheiden lassen, ob er Service nutzen will -> Dies ist wichtig um einen 'Kaufvertrag' abzuschließen (eContracting) 
\subsection{Zusammenfassung}
Um die wirtschaftliche Perspektive abzuschließen, werden im folgenden die Ergebnisse des Kapitels zusammengeführt. Das Business Model Canvas hat ein weitreichenden Blick auf den Software Shop geboten. Anhand der Kundensegmente konnten Aspekte der Kundenbeziehungen und des Marketings identifiziert werden. Vor allem letzteres bedarf weiterer Arbeit. Ein gut geplantes und sinnvolles Marketingkonzept ist wichtig für den Erfolg des Shops. Auch die Nutzenversprechen wurden anhand der Kunden abgeleitet. Aus ihnen werden die Einnahmequellen, als auch die Schlüsselressourcen und -aktivitäten abgeleitet. Die Identifikation der Schlüsselpartner und das darstellen der Kostenstrukturen runden den grundlegenden Blick auf die Aufgaben eines neuen Markts ab.\\

Die Schlüsselaktivitäten des Businessmodels konnten im Rahmen einer Wertschöpfungskette logisch unterteilt und erläutert werden. Hierdurch wurde zum einen der Ablauf verdeutlicht durch welchen eine neue Software zur Aufnahme in den Shop gehen musste, zum anderen wurden auch die nur indirekt mit der Bereitstellung in Verknüpfung stehenden Aufgaben erläutert. Die Wertschöpfungskette, welche die Aufgaben des Software Shops definiert, ist in der Supply Chain als \textbf{Software Shop} dargestellt. Diese zeigt die anderen an der Wertschöpfung beteiligten Schlüsselpartner und in welcher Beziehung diese zueinander stehen.\\

Um den Ablauf eines Einkaufs im Prototypen darzustellen, werden im folgenden technische Konzepte vorgestellt, welche in diesem implementiert werden können. Sie verdeutlichen zum einen den Kommunikationsablauf zwischen Auto und Service Providern im Kontext einer autonomen Parkplatznutzung. Außerdem wird ein Klassifizierungskonzept von Software vorgestellt, welches im Kontext der Eingangslogistik\textit{(Software Klassifizierung)} genutzt werden kann.