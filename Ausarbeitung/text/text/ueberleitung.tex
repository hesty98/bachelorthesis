\section{Überleitung}
\subsection{Eingrenzung des Themenbereichs}
Im Rahmen des Forschungsseminars wurden viele Themenbereiche angeschnitten. Da eine umfassende Erarbeitung all dieser den Rahmen der Bachelorarbeit sprengen würden, wurden in Absprache mit den Betreuern einige Änderungen für das weitere Vorgehen vorgenommen.\\

\textbf{Ausgliederung von User-Centered-Design}\\
Da die Prinzipien des User-Centered-Designs vorwiegend optimierende Faktoren für UX und UI sind, diese Arbeit jedoch einen Prototypen erstellt und kein fertiges Produkt, werden diese außer Acht gelassen. Dies bedeutet nicht, dass sie nicht relevant für diesen Markt sind, eher im Gegenteil. Der Umfang dieser ist zu umfassend um sie ausreichend in der Arbeit zu behandeln.\\

\textbf{Änderung des Anwendungsfalls}\\
Ursprünglich sollte der Prototyp den Absatz einer Software darstellen, welche die autonomen Fahrfunktionen eines Fahrzeugs erweitert. Durch diese Software hätte ein Fahrzeug eine Baustelle selbstständig durchfahren können, ohne das die Fahraufgabe an den Fahrer abgegeben wird. Mit Abschluss des Forschungsseminars wurde sich darauf geeinigt dies zu verwerfen. Der neue Anwendungsfall stellt den Lebenszyklus einer Software dar, welche ein automatischen Parken auf öffentlichen Parkplätzen ermöglicht.\\
Das Fahrzeug kann mit der Schranke bzw. dem Parkautomaten kommunizieren und eine Anfrage für ein Parkticket sowie einen Parkplatz stellen. Anders als im vorherigen Anwendungsfall, benötigt die Software während sie ausgeführt wird noch Nutzereingaben, da bei jeder Interaktion ein Parkticket gekauft wird. hierbei handelt es sich um einen elektronischen Vertragsabschluss, weshalb eine explizite Bestätigung laut Gesetzgeber notwendig ist.\footnote{quelle: eBusiness}\\
Die Unterschiede dieser Anwendungsfälle haben dazu beigetragen, eine Klassifizierung von Softwares anhand von Zugriffsrechten auf die Teilsysteme eines Fahrzeugs vorzunehmen. Dies passiert in Kapitel \ref{technische_konzepte}.\\

Im Rahmen des neuen Anwendungsfalls verlieren im Forschungsseminar beleuchtete Aspekte an direkter Bedeutung für diese Arbeit. Das vorgestellte OpenScenario-Speicherformat wurde verwendet um eine Situation, welche nicht autonom bewältigbar ist aufzuzeichnen und zu speichern. Dieses und der konzipierte SUPR werden im neuen Anwendungsfall nicht benötigt, da dieser keine reine Fahrfunktionssoftware umfasst. Die Bedeutung beider sind insbesondere für das Erkennen und Bereitstellen weiterer autonomer Fahrfunktionen essentiell und sollten daher weiterhin im Hinterkopf behalten werden.\\

\subsection{Weiteres Vorgehen}
Die folgenden Kapitel führen in den Markt neuer Fahrumfänge ein und verdeutlichen wichtige  Bausteine der Wertschöpfungskette. Zunächst wird in Kapitel \ref{wsk} ein Überblick über den Markt zur Erkennung und Bereitstellung neuer Fahrumfänge mittels eines Business-Model-Canvas geschaffen. Anhand der gewonnenen Erkenntnisse werden die wichtigsten Bausteine der Wertschöpfungskette identifiziert, erläutert und miteinander verknüpft. Zusätzlich wird der mögliche Absatzprozess einer Software modelliert.\\\\
Es wird ein Konzept zur Klassifizierung von Software vorgestellt, nach der Art der unterschiedlichen Zugriffsrechte auf die Systeme des Fahrzeugs. Für die Bereitstellung von Software wird ein Kommunikationsprotokoll ausgearbeitet, welches die sichere Interaktion zwischen Server, Fahrzeug und Serviceprovidern ermöglicht.\\\\
In Kapitel \ref{prototyp} wird der Prototyp vorgestellt, welcher die Ergebnisse voriger Kapitel zusammenstellt und einen verdeutlicht. Hierzu wird zunächst die Architektur des Prototypen vorgestellt und verdeutlicht, an welcher Stelle die jeweiligen Konzepte aus dem Forschungsseminar integriert werden. Durch das befolgen der Anleitung im darauffolgenden Kapitel kann der Prototyp auf dem eigenen PC installiert werden. Anschließend werden die wesentlichen Funktionen des Prototypen verdeutlicht und erklärt, wie man diesen steuert. Abschließend erfolgen ein Ausblick für die Weiterentwicklung des Prototypen und eine Diskussion als Fazit der Arbeit.