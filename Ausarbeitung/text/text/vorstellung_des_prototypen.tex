\section{Vorstellung des Prototypen}\label{prototyp}
Um die Mehrwerte der Wertschöpfungskette zu verdeutlichen wird ein Prototyp implementiert. Dieser soll den Prozess der Installation, des Updaten und des Nutzen von Softwarepaketen ersichtlich werden lassen und ein technisches Konzept zur Umsetzung der Wertschöpfungskette bereitstellen. Hierzu wird der Absatzprozess abschließend exemplarisch im Kontext von öffentlichen Parkplätzen vorstellt. \textit{Kapitel x\footnote{einfügen}}
\subsection{Architektur des Prototypen}

\subsection{Installation}
\begin{itemize}
	\item MMS installieren
	\item Carla optional
	\item ccu ausführen
\end{itemize}
\subsection{Funktionen und Nutzung des Prototypen}
\begin{itemize}
	\item Stages der Simulation (Schritt für Schritt durch den Absatzprozess)
	\item Wie ich von Stage zu Stage komme
	\item stages: no registered car, car in perception area, car leaving, car installing sw, car using service
\end{itemize}
%Sind alle Teilsysteme installiert \textit{(Carla-Installation ist optional)} kann der Anwendungsfall gestartet werden. Im ersten Schritt wird dem Auto signalisiert, dass es in der Simulation losfahren kann. Hierzu muss der "Szenario Starten"-Knopf gedrückt werden. 
\subsection{Analyse der Prototypen und Ausblicke der Weiterentwicklung}
- Mehrwerte aufzählen\\
   Software nutzen auf Autos, welche dem Auto Fahrbefehle gibt.
- Vorteile der gewählten Architektur\\
- Nachteile der gewählten Architektur \& Umsetzung.
