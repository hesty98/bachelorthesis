\subsection{Business Model Canvas} \label{bmc}
Das 2004 von Alexander Osterwalder entwickelte Busines Model Canvas \textit{(BMC)} schafft einen Überblick über die Aufgaben, die Kosten- und Partnerstrukturen sowie den Kundensegmenten eines Unternehmens. Es hilft den Fokus auf die wesentlichen Zielsetzungen dessen zu setzen.\cite[Vgl. ]{b105} Folgend werden die Kundensegmente, die Kundenbeziehungen, die Marketingkanäle und die Einnahmequellen betrachtet anhand welcher anschließend die Nutzenversprechen sowie die Schlüsselressourcen und -aktivitäten eines Shops bestimmt werden. Durch die abschließende Bestimmung von Schlüsselpartnern und der möglichen Kostenstruktur eines Softwareshops wurden \glqq alle wesentlichen Elemente eines Geschäftsmodells in ein skalierbares System gebracht\grqq\cite[S.\, 14]{bmc}, anhand wessen die Bausteine der Wertschöpfungskette abgeleitet werden können.

\subsubsection{Kundensegmente}
Die Kundensegmente eines Shops lassen sich in Einzelkunden und Flottenbetreiber unterteilen, wobei sich Flottenbetreiber in zwei weitere Segmente aufteilen lassen: \textit{"Leihe und Leasing"} umfasst Autovermieter, Unternehmen die ihren Mitarbeitern Leasingwagen bereitstellen, aber auch weitere wie Car-Sharing Unternehmen. Deren Kunden und Mitarbeiter erwarten eine grundlegende Sammlung an Software im Mietfahrzeug vorzufinden und wollen möglicherweise selbstständig weitere Softwares auf eigene Rechnung installieren können. Neben \glqq Leihe und Leasing\grqq sind auch Unternehmen mit Firmenwagen wie zum Beispiel Lieferdienste, Taxi-Unternehmer oder Bauunternehmen ein gesondertes Kundensegment. Beide haben kleine oder große Fahrzeugflotten und müssen Softwarekäufe dementsprechend skalieren können. \\
Einzelkunden sind die übliche Autofahrer, die ein privates Fahrzeug besitzen und Software auf diesem Installieren möchten. Durch die enorme Größe ist es sinnvoll dieses Segment weiter zu unterteilen.
%
%\begin{itemize}
%	\item[\textbf{18-25}] 
%	Die 18-25 Jährigen sind mit modernen Technologien wie dem Computer und dem Smartphone aufgewachsen. Sie stellt die Gruppe mit dem durchschnittlich geringsten Einkommen dar. Für sie ist es intuitiv zum Handy, Computer oder anderen Alltags-unterstützenden Technologien zu greifen. Sie erkennen die potentiellen Mehrwerte von Technologien leichter als ältere Segmente und neigen daher vermutlich eher zum Kauf von Software. 
%	
%	\item[\textbf{25-33}]
%	Ebenfalls bestens mit Technik vertraut, umfasst dieses Segment die vermutlich wichtigsten Kunden. Es umfasst viele verdienende Menschen, die am Anfang ihrer Karriere stehen und dabei sind sich ein Leben aufzubauen. Sie sind in der Lage mehr Software als die jüngeren Segmente zu Kaufen. Durch ihr fortgeschrittenes Alter sind sie für ältere Segmente oft ein Ansprechpartner im Bezug auf technologische Fragen.\cite{b102}
%		
%	\item[\textbf{33-50}]
%	Fest im Leben stehend, stellt dieses Segment das Mengenmäßig größte dar.\cite[S. 4]{fahrerAlter} Im Gegenteil zu den jüngeren Segmenten ist hier wahrscheinlich dass die meisten ein eigenes Fahrzeug haben, auf welchem Sie Software installieren können. 
%	
%	\item[\textbf{50-65}]
%	Auch in diesem Segment haben die meisten ein eigenes Auto.\cite[ebenda]{fahrerAlter} Dieses wird sich öfters geteilt, da die Notwendigkeit für Zwei Autos nicht mehr gegeben ist. Die Anforderungen sind vergleichbar zu denen der 33-50 Jährigen, jedoch lässt sich diese Gruppe im Bezug auf neue Technik eher beraten.
%	
%	\item[\textbf{65-75}]
%	Je älter der Kunde ist, desto geringer ist die durchschnittliche Technikaffinität. In diesem Segment sind die Mehrwerte von Technik maßgebend dafür, ob ein Kauf stattfindet oder nicht. Eröffnet sich Raum zur Kritik, neigen diese eher vom Kauf ab.\cite{b101} Zeitgleich sind sie von den jüngeren Segmenten einfach zu beeinflussen wenn es um den Kauf neuer Technik geht.
%	
%	\item[\textbf{75 +}]
%	Durch das fortgeschrittene Alter benötigt dieses Segment Unterstützung bei der Autofahrt. Es legt Wert auf ein weitgehend selbstständig fahrendes Fahrzeug, da so Strecken zurückgelegt werden können die im Normalfall nicht bewältigt worden wären. 
%\end{itemize}

Sowohl Einzelkunden als auch Flottenbetreiber haben ähnliche Anforderungen an einen Software Shop. Durch neue Softwares soll ein Fahrzeug vermehrt selbstständig fahren können und den Insassen so Zeit zu sparen. Kunden sollten akquirierte Softwares verwalten und überwachen können, um so einen Überblick ihres Fahrzeugs zu haben.

\subsubsection{Kundenbeziehungen}
%Obwohl das Einzelkundensegment quantitativ größer ist, haben alle Kundensegmente die gleiche Wichtigkeit.
Um die Kunden nach dem ersten Kauf nicht zu Verlieren, muss eine positive Bindung zwischen ihnen und dem Shop aufgebaut werden. So sollte die erste Software, die dem Kunden vorgeschlagen wird einen deutlichen Mehrwert für diesen bieten. Hierdurch steigt die Zufriedenheit des Kunden und ein erneuter Kauf ist wahrscheinlicher.\\
Um langfristig viel Software über den Shop absetzen zu können, ist auch darüber hinaus eine gute Kundenbeziehung wichtig. Fahrzeughalter müssen \textit{\textbf{dem Shop vertrauen}} können und bei \textit{\textbf{Kaufentscheidungen unterstützt}} und \textbf{\textit{beraten}} werden. Das Vertrauen kann gesteigert werden indem akquirierte Softwares deutliche Mehrwerte für Fahrzeughalter bieten. Auch die Einbeziehung von Fahrzeughaltern in die Entwicklung von Softwares kann eine postive Kundenbeziehung fördern. Um Flottenbetreiber beim Kauf von Software zu unterstützen, ist eine Web-App Sinnvoll, über die für eine große Menge an Fahrzeugen Softwares gekauft, verwaltet und überwacht werden können.

\subsubsection{Marketingkanäle} 
Es ist davon auszugehen, dass ab dem Zeitpunkt des Markteintritts hergestellte Fahrzeuge von Partnerunternehmen den Shop vorinstalliert haben. Fahrzeughalter können daher umgehend neue Softwares kaufen und sollte über diesen Umstand möglichst häufig informiert werden. Hierzu sollten Werbekampagnen auf so vielen Marketingkanälen wie möglichen erfolgen, damit der Shop und die durch ihn geschaffenen Mehrwerte allgegenwärtig in der Gesellschaft werden. Zu Anfang könnten Rabatte oder weitere Sonderaktionen neue Kunden anlocken und vom Kauf überzeugen. In Zeiten von Social Media ist das Influencer Marketing wichtig, da diese Plattformen alleine in Deutschland täglich für ca. 140 Minuten genutzt werden.\cite[Vgl. ]{socialmedia}

\subsubsection{Nutzenversprechen} \label{nv}
Nutzenversprechen sind die Werteversprechen eines Unternehmens, die bestimmte Bedürfnisse von Kunden abdecken. Sie können von Personas, Interviews, Umfragen oder anderen abgeleitet werden.\\\\
\textbf{1. Orts- und Zeitunabhängige Akquise eigens ausgewählter Softwares}\\
Softwares sollen von Fahrzeughaltern orts- und zeitunabhängig heruntergeladen werden können. Die Softwares können selber ausgewählt werden, wodurch nur tatsächlich benötigte Softwares installiert und einige Fahrten zum Mechaniker verhindert werden.\\\\
\textbf{2. Überwachung, Verwaltung und Vorschlagen von Software}\\
Fahrzeughalter sollen überwachen können, welche installierten Softwares vom Fahrzeug wie oft genutzt werden. Hierdurch können nicht benötigte Softwares identifiziert, deinstalliert und anschließend möglicherweise im Shop bewertet werden. Neben dieser Unterstützung, wird auch die Suche nach möglicherweise passenden Softwares für den Fahrzeughalter automatisiert. Dies kann vor allem weniger technikaffine Kundensegmente beim Kauf unterstützen und Sicherheit im Umgang mit dem Fahrzeug schaffen. Wie der Bedarf einer Software erkannt werden kann, wird in den Kapiteln \ref{2.3} und \ref{umgebungssuche} erläutert.\\\\
\textbf{3. Autofahren wird sicherer}\\
Autonom fahrende Fahrzeuge könnten den Straßenverkehr sicherer zu machen, da mittels Sensoren und Recheneinheiten ein Fahrzeug mehr Daten aufnehmen und verarbeiten kann als der Mensch.\cite[Vgl. ]{b103} Neben den Sicherheitsvorteilen die einzelne Softwares für ein Fahrzeugs haben können, ist weitblickend vor allem der Einsatz von V2X-Kommunikation wertvoll, da diese die Sicherheit des Straßenverkehrs entscheidend verbessern kann werden.\cite[Vgl.]{bmd}\cite[Vgl. S. 19]{vda}\\\\
\textbf{4. Lebenszeit des Autos wird verlängert}\\
Aufgrund der steigenden Sicherheit ist es wahrscheinlich, dass Fahrzeuge künftig unfallfreier Fahren können\footnote{quelle Waymo}. Durch den stetigen Kauf neuer und geregelten Updates bereits gekaufter Softwares bleiben die Softwares von Fahrzeugen länger aktuell und können so die durchschnittliche Lebenszeit eines Fahrzeugs verlängern, was die Anschaffung eines neuen Fahrzeugs aufschieben kann.\\\\
\textbf{5. Geringerer Wertverlust von Fahrzeugen}\\
Durch das eigenständige erweitern des Fahrumfangs der Fahrzeugs kann der Wert steigen. Nicht nur steigert Preis der Software den Wert, sondern auch die bereits vorhandene Konfiguration der Softwares beinhaltet einen Wert an sich.\\\\
\textbf{6. Fahrzeughalter gewinnen Zeit}\\
Durch die Integration neuer Fahrfunktionen können zunehmend mehr Situationen des Straßenverkehrs zurückgelegt werden ohne dass der Fahrer selber die Steuerung übernehmen muss. Die hierdurch gewonnene Zeit kann von den Insassen zum Arbeiten, gemeinsamen Interaktionen oder anderweitig genutzt werden. Durch diese gewonnene Zeit kann eine Autofahrt künftig weniger anstrengend sondern eher wertvoll für alle sein.\\\\
\textbf{7. Stetige Erweiterung des Softwareangebots}\\
Die Bereitstellung eines Softwareshops ist damit gekoppelt, dass eigene und externe Entwicklerteams stetig neue Softwares veröffentlichen und bestehende weiterentwickeln. Die Weltweit ca. 1,3 Milliarden registrierten Kraftfahrzeuge \cite[Vgl.]{b106} bieten eine sehr große Kundenbasis und somit auch eine gute Einnahmequelle, welche viele Entwicklerteams anlocken können. Das Spektrum neuer Softwares kann durch die steigende Anzahl an Entwicklern stark zunehmen und Anwendungsfälle können schneller abgedeckt werden.\\\\
\textbf{8. Interaktion mit der Autoumwelt}\\
Bestimmte Softwares können die Kommunikation mit anderen Akteuren der Fahrzeugumwelt \textit{(Ampeln, Parkplätze, andere Fahrzeuge, Drive-In-Restaurants, uvm.)} ermöglichen und so neue Möglichkeiten zur Interaktion untereinander schaffen und auch die Effizienz des Verkehrs nachhaltig durch V2X-Kommunikation steigern. \cite[Vgl. S.19]{vda}

\subsubsection{Einnahmequellen}
Ein Softwareshop hat zwei unterschiedliche Einkommensströme. Zum einen kann es eine 'Anmeldegebühr' geben, die Softwareprovider zahlen müssen um Software im Shop veröffentlichen zu können, wie es auch für den Google Play Store der Fall ist. Außerdem können diese ihre Softwares im Shop bewerben, indem sie Werbeflächen kaufen auf welchen ihre Softwares zum Kauf hervorgehoben werden. Dies ist Vergleichbar mit dem Ergebnis einer Google-Suche, bei der ganz oben \textit{gesponserte} Internetseiten/Produkte zu finden sind.\\
Neben Einnahmen durch die Softwareprovider wird auch Umsatz durch den Verkauf und die Nutzung von Software generiert. Beim Kauf einer Software geht ein fester Prozentsatz des Verkaufspreises an den Shop Betreiber zurück. Zum Vergleich: Bei Android Apps liegt der Anteil den Google einnimmt bei 30\%. Auch bei entstehenden In-App-Käufen \textit{(z.B. für die Nutzung eines \textbf{Services})} geht ein Anteil von 10\% an Google.\\
Durch die Entwicklung von Entwicklungs- und Analyse-Tools können weitere Einnahmen generiert werden. Software Provider können diese kaufen und nutzen, wodurch sie bei der Entwicklung neuer Softwares unterstützt werden.

\subsubsection{Schlüsselressourcen}
Die wichtigsten Ressourcen sind die \textbf{geschaffene Plattform des Softwareshops} und die dort vertriebenen \textbf{Softwares}. Um eine große Menge an Softwares und sonstigen Daten von Fahrzeugen speichern und analysieren zu können, ist vor allem ein starkes und effizientes Backend ist wichtig, welches zugleich für die Suche als auch für die Verteilung von Software zuständig ist. Damit der Faktor der Ortsunabhängigkeitrealisiert werden kann, ist ein stabiles Kommunikationsnetzwerk (z.B. 5G\cite[S. 10]{vda}) notwendig. \\
Da Softwares zum Großteil von Software Providern bereitgestellt werden und der Shop ohne das stetige hinzufügen und aktualisieren von Softwares an Attraktivität verliert ist eine gute Beziehung zu diesen essentiell.
\subsubsection{Schlüsselaktivitäten}\label{key_activities}
Die Schlüsselaktivitäten sind die Aufgaben des Unternehmens, die zur Erfüllung vorgestellter Nutzenversprechen führen. Sie sind an die Schlüsselressourcen und Nutzenversprechen gekoppelt und sollen die Anforderungen dieser erfüllen. Die Schlüsselaktivitäten stellen im Kontext einer Wertschöpfungskette \textbf{die wichtigsten Bausteine dieser} dar. Zunächst werden sie daher in vier Gruppen unterteilt, welche die Rolle der jeweiligen Bausteine in der Wertschöpfungskette verdeutlichen sollen. In Kapitel \ref{wsk} werden sie näher erläutert und in einen logischen Kontext gebracht.\\\\
\textbf{1. Aktivitäten der Bedarfserkennung von Software}
\begin{itemize}
	\item Software Klassifizierung
	\item Automatische Erkennung von Softwarebedarf
\end{itemize}
\vspace{0.2cm}
\textbf{2. Aktivitäten der Bereitstellung von Software}
\begin{itemize}
	\item Sicherheitsverifikation von Software anhand von Sicherheitskonzepten
	\item Sicherer Download über das Internet 
	\item Anbindung von elektronischen Zahlungsschnittstellen
	\item Angebotsunterbreitung
	\item Eigenständige Softwareentwicklung
\end{itemize}
\vspace{0.2cm}
\textbf{3. Aktivitäten des generellen Betriebs}
\begin{itemize}
	\item Shop Verwaltung
	\item Shop \textit{(Weiter-)}Entwicklung und Wartung
	\item Eigenständige Verwaltung von Software durch den Kunden
	\item Eigenständige Überwachung von Software durch den Kunden
	\item Kundenservice und individuelle Beratung
\end{itemize}
\vspace{0.2cm}
\textbf{4. Aktivitäten zur Verbesserung der Supply Chain}
\begin{itemize}
	\item IDE\textit{(Entwicklungsumgebung)} entwickeln
	\item Dokumentation \& API Reference
	\item Tool Entwicklung \textit{(Für Entwickler)}
\end{itemize}

\subsubsection{Schlüsselpartner}
Die Schlüsselpartner sind die Teilnehmer der Supply Chain, von denen der Software Shop direkt abhängig ist.\cite[Vgl. ]{b105} Eine gute Beziehung zu Ihnen ist wichtig, um die aufgestellten Nutzenverprechen erfüllen zu können.
\begin{itemize}
	\item \textbf{Software Provider}\\
	Software Provider stellen die Softwares her, die im Shop zur Verfügung gestellt werden und sind daher die wichtigsten Lieferanten des Shops. Sie sollten bei der Entwicklung von Softwares unterstützt werden, da ohne Softwares kein Shop möglich ist. Die Entwickelten Softwares dürfen \textbf{keine Sicherheitslücken enthalten} und sollten für möglichst viele Automarken und -modelle verfügbar sein. Die Einbeziehung von ihnen in den Entwicklungsprozess des Shops und von Software APIs kann ihr Verständnis und ihre Fähigkeiten stärken und somit zum Erfolg des Shops beitragen.
	
	\item \textbf{Automobilhersteller}\\
	Damit die Entwickelten Softwares auf Fahrzeugen installiert werden können, müssen Automobilhersteller als Partner gewonnen werden damit diese den Shop in ihre Fahrzeuge integrieren. Je mehr Automobilhersteller \textit{(OEMs)}an dem Shop beteiligt sind desto größer ist dessen Reichweite, wodurch auch die Attraktivität dessen für neue Software Provider steigt. Die Einbeziehung möglichst vieler OEMs soll verhindern, dass ein Konkurrenzprodukt des Software Shops entsteht.\\
	Integriert ein OEM den Shop in seine Fahrzeuge, muss dieser die Entwicklung von APIs für die eigenen Fahrzeugmodelle vorantreiben, damit Softwares auf diesen ausgeführt werden können.
	
	\item \textbf{Mobilfunkbetreiber}\\
	Damit Softwares orts- und zeitunabhängig Heruntergeladen werden können, müssen Autos eine kontinuierliche Verbindung zum Internet haben. Wenn Fahrzeuge jedoch nicht in der eigenen Garage stehen ist es eher unwahrscheinlich, dass sie in mit einem WLAN verbunden sind. Durch ein stabiles und gut ausgebautes mobiles Datennetz kann ein Fahrzeug auch ohne die Verbindung mit einem WLAN oder anderen Netzwerkstrukturen Softwares herunterladen. Damit die Kosten für Fahrzeughalter hierbei skalierbar gehalten werden, ist eine Partnerschaft mit Mobilfunkanbietern sinnvoll mittels welcher der Internetzugriff für Softwaredownloads günstiger Werden kann.
	
	\item \textbf{Elektronische Bezahlsysteme}\\
	Um den Kauf von Softwares einfach zu gestalten, ist die Einbindung elektronischer Bezahlsysteme wie Paypal, MasterCard, Klarna oder sonstigen wichtig. Hierdurch kann der Kauf von Softwares schnell als auch orts- und zeitunabhängig abgeschlossen werden.

	\item \textbf{Service Provider}\\
	Von Service Providern ist der Shop zwar nicht abhängig, aber durch sie können viele  Mehrwerte für Fahrzeughalter geschaffen werden die sonst nicht existieren würden. Sie bieten Services an, die von Fahrzeughaltern mit Hilfe einer Software in Anspruch genommen und genutzt werden können.
\end{itemize}

\subsubsection{Kostenstruktur}
Die Pflege der Beziehungen mit den Schlüsselpartnern ist wichtig, da der Software Shop ohne sie nicht existieren könnte. So kann die Entwicklung einer IDE, von Tutorials, API Referenzen und Analysetools für Software Provider zugleich wichtig als auch kostspielig sein. Auch sonstige Events des Unternehmens wie Konferenzen und Tagungen, welche zur Bindung von Schlüsselpartnern dienen, sind mit hohen Kosten verbunden.\\
Damit die Integration des Shops in die Fahrzeuge von Automobilherstellern erfolgt ist es wahrscheinlich, dass Automobilhersteller prozentuale Gewinne des Shops erhalten. Diese Summe könnte anhand der Menge verkaufter Softwares auf Fahrzeugen dieser Marke berechnet werden. Weiterhin fallen Kosten für den Betrieb , die (Weiter-)Entwicklung und die Vermarktung des Shops an. Diese sind im wesentlichen für die Bezahlung von Managern, Entwicklern und Designern. Weiter fallen Stromkosten für das Betreiben von Servern an.\\\\\\\\\\
Das Business Model Canvas hat einen Überblick über die wesentliche Aspekte einer Unternehmung geboten, welche neue Fahrfunktionen für intelligente Fahrzeuge bereitstellen möchte. Es bietet Ansätze für Umsetzung in der realen Welt und liefert wichtige Erkenntnisse für die im folgenden aufgestellte Wertschöpfungskette.