
\subsection{Business Model Canvas} \label{bmc}
Das 2004 von Alexander Osterwalder entwickelte Busines Model Canvas \textit{(BMC)} schafft einen Überblick über die Aufgaben, die Kosten- und Partnerstrukturen sowie den Kundensegmenten eines Unternehmens. Es hilft den Fokus auf die wesentlichen Zielsetzungen dessen zu setzen.\footnote{https://ut11.net/de/blog/dein-geschaftsmodell-kompakt-der-business-model-canvas/} Folgend werden die Kundensegmente, die Kundenbeziehungen, die Marketingkanäle und die Einnahmequellen betrachtet. Anhand dieser werden anschließend die Nutzenversprechen sowie die Schlüsselressourcen und -aktivitäten bestimmt. Durch die abschließende Bestimmung von Schlüsselpartnern und der möglichen Kostenstruktur eines Softwareshops wurden "alle wesentlichen Elemente eines Geschäftsmodells in ein skalierbares System gebracht."\footnote{https://www.startplatz.de/startup-wiki/business-model-canvas/}

\subsubsection{Kundensegmente}
Die Kundensegmente des Marktes lassen sich in Einzelkunden, Gruppen und Flottenbetreiber aufteilen. Flottenbetreiber lassen sich in zwei Segmente aufteilen: \textit{"Leihe und Leasing"} umfasst Autovermieter, Unternehmen die ihren Mitarbeitern Leasingwagen bereitstellen, aber auch weitere wie Car-Sharing Unternehmen. Deren Kunden erwarten eine grundlegende Sammlung an Software im Mietfahrzeug vorzufinden und wollen selbstständig weitere Softwares auf eigene Rechnung installieren können. Neben Leihe und Leasing sind auch Unternehmen mit Firmenwagen ein gesondertes Kundensegment. Beide haben kleine oder große Fahrzeugflotten und müssen Softwarekäufe dementsprechend skalieren können. \\
Einzelkunden sind die übliche Autofahrer, die ein privates Fahrzeug besitzen und Software auf diesem Installieren möchten. Gruppen sind mehrere Einzelkunden, die gemeinsam Software kaufen um hierdurch Kosten zu sparen. Hierdurch werden Familienkäufe aber auch Käufe mit Freunden möglich. Durch ihre enorme Größe ist es sinnvoll, auch dieses Segment weiter zu unterteilen.

\begin{itemize}
	\item[\textbf{18-25}] 
	Die 18-25 Jährigen sind mit modernen Technologien wie dem Computer und dem Smartphone aufgewachsen. Sie stellt die Gruppe mit dem durchschnittlich geringsten Einkommen dar. Für sie ist es intuitiv zum Handy, Computer oder anderen Alltags-unterstützenden Technologien zu greifen. Sie erkennen die potentiellen Mehrwerte von Technologien leichter als ältere Segmente und neigen daher vermutlich eher zum Kauf von Software. 
	
	\item[\textbf{25-33}]
	Ebenfalls bestens mit Technik vertraut, umfasst dieses Segment die vermutlich wichtigsten Kunden. Es umfasst viele verdienende Menschen, die am Anfang ihrer Karriere stehen und dabei sind sich ein Leben aufzubauen. Sie sind in der Lage mehr Software als die jüngeren Segmente zu Kaufen. Durch ihr fortgeschrittenes Alter sind sie für ältere Segmente oft ein Ansprechpartner im Bezug auf technologische Fragen.\footnote{quelle}
		
	\item[\textbf{33-50}]
	Fest im Leben stehend, stellt dieses Segment das Mengenmäßig größte dar.\footnote{quelle} Im Gegenteil zu den jüngeren Segmenten ist hier wahrscheinlich dass die meisten ein eigenes Fahrzeug haben, auf welchem Sie Software installieren können. 
	
	\item[\textbf{50-65}]
	Auch in diesem Segment haben die meisten ein eigenes Auto.\footnote{quelle} Dieses wird sich öfters geteilt, da die Notwendigkeit für Zwei Autos nicht mehr gegeben ist. Die Anforderungen sind vergleichbar zu denen der 33-50 Jährigen, jedoch lässt sich diese Gruppe im Bezug auf neue Technik eher beraten.
	
	\item[\textbf{65-75}]
	Je älter der Kunde ist, desto geringer ist die durchschnittliche Technikaffinität.\footnote{quelle} In diesem Segment sind die Mehrwerte von Technik maßgebend dafür, ob ein Kauf stattfindet oder nicht. Eröffnet sich Raum zur Kritik, neigen diese eher vom Kauf ab.\footnote{quelle} Zeitgleich sind sie von den jüngeren Segmenten einfach zu beeinflussen wenn es um den Kauf neuer Technik geht. \footnote{quelle}
	
	\item[\textbf{75 +}]
	Durch das fortgeschrittene Alter benötigt dieses Segment Unterstützung bei der Autofahrt. Es legt Wert auf ein weitgehend selbstständig fahrendes Fahrzeug, da so Strecken zurückgelegt werden können die im Normalfall nicht bewältigt worden wären. 
\end{itemize}

Sowohl Einzelkunden als auch Flottenbetreiber haben ähnliche Anforderungen an einen Software Shop. Durch neue Softwares soll das Fahrzeug vermehrt selbstständig fahren, um den Insassen Zeit zu ersparen. Kunden wollen die akquirierten Softwares verwalten und überwachen können, um so einen Überblick ihres Fahrzeugs zu haben.

\subsubsection{Kundenbeziehungen}
Obwohl das Einzelkundensegment quantitativ größer ist, haben alle Kundensegmente die gleiche Wichtigkeit. Um die Kunden nach dem ersten Kauf nicht zu Verlieren, muss eine positive Bindung zwischen ihnen und dem Shop aufgebaut werden. So sollte die erste Software, die dem Kunden vorgeschlagen wird einen deutlichen Mehrwert für diesen bieten. Hierdurch steigt die Zufriedenheit des Kunden und ein erneuter Kauf ist wahrscheinlicher.\\
Um langfristig viel Software über den Shop absetzen zu können, ist auch darüber hinaus eine gute Kundenbeziehung wichtig. Fahrzeughalter müssen \textit{\textbf{dem Shop vertrauen}} können, bei \textit{\textbf{Kaufentscheidungen unterstützt}} und \textbf{\textit{beraten}} werden. Diese Ziele können erreicht werden, indem akquirierte Softwares deutliche Mehrwerte für Fahrzeughalter bieten. Auch die Einbeziehung von Fahrzeughaltern in die Entwicklung von Softwares kann eine postive Kundenbeziehung aufbauen. Um Flottenbetreiber beim Kauf von Software zu unterstützen, ist eine Web-App Sinnvoll, über die für eine große Menge an Fahrzeugen die gleichen Softwares gekauft, verwaltet und überwacht werden können.

\subsubsection{Marketingkanäle}
Um eine marktführende Position zu erreichen, sollten bei Markteintritt Werbekampagnen auf allen möglichen Marketingkanälen stattfinden. Die Altersgruppen des Einzelkundensegments werden über unterschiedliche Kanäle erreicht. Um ältere Segmente (50+) zu erreichen, ist das Nutzen \textit{traditioneller} Medien wie Plakate, Zeitschrift oder dem Fernsehen Sinnvoll. Zu Sendezeiten, in denen überwiegend junge Zuschauer aktiv sind, sollen die Werbespots dementsprechend abgeändert werden. Jüngere Kunden sind aktiver auf anderen Medien oder Plattformen, wie Instagramm, Facebook, Twitter, aber auch und vor allem Youtube. Über diese kann zum einen Influencermarketing \textit{(Influencer bewerben die Plattform)} betrieben werden oder auch klassische Werbeanzeigen geschaltet werden. Ziel ist es, dass viel über den neuen Shop geredet und berichtet wird, damit dieser allgegenwärtig wird.

\subsubsection{Nutzenversprechen} \label{nv}
\textbf{1. Orts- und Zeitunabhängige Akquise eigens ausgewählter Softwares}\\
Softwares sollen von Fahrzeughaltern orts- und zeitunabhängig heruntergeladen werden können. Die Softwares können selber ausgewählt werden. Hierdurch werden die Fahrten zum Mechaniker aufgrund von Softwareupdates gespart.\\\\
\textbf{2. Überwachung, Verwaltung und Vorschlagen von Software}\\
Fahrzeughalter sollen überwachen können, welche installierten Softwares wie oft genutzt werden. Hierdurch können nicht benötigte Softwares identifiziert und deinstalliert werden. Neben dieser Unterstützung, wird auch die Suche nach möglicherweise passenden Softwares für den Fahrzeughalter automatisiert. Dies kann vor allem weniger technikaffine Kundensegmente beim Kauf unterstützen und Sicherheit im Umgang mit dem Fahrzeug schaffen. Wie der Bedarf einer Software erkannt werden kann, wird in Kapitel \ref{2.3} erläutert.\\\\
\textbf{3. Autofahren wird sicherer}\\
Autonom fahrende Fahrzeuge sind in der Lage, den Straßenverkehr sicherer zu machen.\footnote{quelle} Hierdurch sinkt die Anzahl an Unfällen und der Verkehrsfluss wird besser.\footnote{quelle} Neben den Sicherheitsvorteilen die einzelne Softwares für ein Fahrzeugs haben können, ist weitblickend vor allem der Einsatz von V2X-Kommunikation wichtig. Durch diese wird die Sicherheit des Straßenverkehrs entscheidend verbessert.\footnote{quelle}\\\\
\textbf{4. Lebenszeit des Autos wird verlängert}\\
Aufgrund der steigenden Sicherheit werden Fahrzeuge künftig unfallfreier Fahren können. Außerdem wird die Fahraufgabe von einer Software schonender für Getriebe- und Motorteile durchgeführt. Durch den stetigen Kauf neuer Softwares und geregelten Updates bereits gekaufter Produkte kann die durchschnittliche Lebenszeit eines Fahrzeugs verlängern, was auch die Anschaffung von Neuwagen Fahrzeugs aufschieben kann.\\\\
\textbf{5. Geringerer Wertverlust von Fahrzeugen}\\
Durch das eigenständige erweitern des Fahrumfangs der Fahrzeugs kann der Wert steigen. Nicht nur steigert Preis der Software den Wert, sondern auch die bereits vorhandene Konfiguration der Softwares beinhaltet einen Wert an sich.\\\\
\textbf{6. Fahrzeughalter gewinnen Zeit}\\
Durch das Nutzen autonomer Fahrfunktionen können immer weitere Strecken zurückgelegt werden ohne dass der Fahrer selber die Steuerung übernehmen muss. Die hierdurch gewonnene Zeit kann zum Arbeiten oder auch anderweitig genutzt werden. Durch diese mögliche gemeinsame Zeit kann eine Autofahrt künftig weniger anstrengend sondern eher wertvoll sein.\\\\
\textbf{7. Stetige Erweiterung des Softwareangebots}\\
Die Bereitstellung eines offenen Markts für Softwareanbieter kann viele Entwickler anlocken, da der Umsatz auf diesem Markt sehr hoch sein könnte \textit{(1 Million verkauft für 20€/Stück = 20 Millionen Umsatz)}. Das Spektrum neuer Softwares kann durch die steigende Anzahl an Entwicklern zunehmend größer werden. Hierdurch können viele Anwendungsfälle schneller abgedeckt werden.\\\\
\textbf{8. Interaktion mit der Autoumwelt}\\
Bestimmte Softwares können die Kommunikation mit anderen Akteuren der Fahrzeugumwelt \textit{(Ampeln, Parkplätze, andere Fahrzeuge, Drive-In-Restaurants, uvm.)} ermöglichen und so neue Möglichkeiten zur Interaktion untereinander schaffen.

\subsubsection{Einnahmequellen}
Der Softwareshop hat zwei unterschiedliche Einkommensströme. Zum einen kann es eine 'Anmeldegebühr' geben, die Softwareprovider zahlen müssen um Software im Shop veröffentlichen zu können. Außerdem können diese ihre Softwares im Shop bewerben, indem sie Werbeflächen kaufen wodurch ihre Software zum Kauf hervorgehoben wird. Dies ist Vergleichbar mit dem Ergebnis einer Google-Suche, bei der ganz oben \textit{gesponserte} Internetseiten/Produkte zu finden sind.\\
Neben Einnahmen durch die Softwareprovider wird auch Umsatz durch den Verkauf und die Nutzung von Software generiert. Beim Kauf einer Software geht ein fester Prozentsatz des Verkaufspreises an den Shop Betreiber zurück. Zum Vergleich: Bei Android Apps liegt der Anteil den Google einnimmt bei 30\%. Auch bei entstehenden In-App-Käufen \textit{(z.B. für die Nutzung eines \textbf{Services})} geht ein Anteil von 10\% an Google.\\
Durch die Entwicklung von Entwicklungs- und Analyse-Tools können weitere Einnahmen generiert werden. Provider können diese kaufen und nutzen, wodurch sie bei der Entwicklung neuer Softwares unterstützt werden.

\subsubsection{Schlüsselressourcen}
Die wichtigsten Ressourcen sind die \textbf{geschaffene Plattform des Softwareshops} und die dort vertriebenen \textbf{Softwares}. Um eine große Menge an Softwares und sonstigen Daten von Fahrzeugen speichern und analysieren zu können, ist vor allem ein starkes und effizientes Backend ist wichtig, welches zugleich für die Suche und Verteilung von Software zuständig ist. Damit der Faktor der Ortsunabhängigkeit von Softwaredownloads realisiert werden kann, ist ein stabiles Kommunikationsnetzwerk (z.B. 5G\footnote{quelle, wo drin steht, dass 5G das tatsächlich lößt}) notwendig. Softwares werden zum Großteil von Software Providern bereitgestellt. Ohne das stetige hinzufügen und aktualisieren von Softwares, verliert der Shop auf Dauer Attraktivität.\\
Eine weitere Schlüsselressource ist die Einbindung eines elektronischen Zahlungsservices, damit Fahrzeughalter Softwares tatsächlich bezahlen können.\\

\subsubsection{Schlüsselaktivitäten}\label{key_activities}
Die Schlüsselaktivitäten sind die Aufgaben des Unternehmens, die zur Erfüllung vorgestellter Nutzenversprechen führen. Sie sind an die Schlüsselressourcen und Nutzenversprechen gekoppelt und sollen die Anforderungen dieser erfüllen. Die Schlüsselaktivitäten stellen im Kontext einer Wertschöpfungskette \textbf{die wichtigsten Bausteine dieser} dar. Daher werden sie im folgenden zunächst in vier Gruppen unterteilt um in Kapitel \ref{wsk} näher erläutert und in einen logischen Kontext gebracht zu werden.\\\\
\textbf{1. Aktivitäten der Bedarfserkennung von Software}
\begin{itemize}
	\item Software Klassifizierung
	\item Automatische Erkennung möglichen Softwarebedarfs
\end{itemize}
\vspace{0.2cm}
\textbf{2. Aktivitäten der Bereitstellung von Software}
\begin{itemize}
	\item Sicherheitsverifikation von Software anhand von Sicherheitskonzepten
	\item Sicherer Download über das Internet bereitstellen
	\item Bereitstellung einer elektronischen Zahlungsschnittstelle
	\item Angebotsunterbreitung
	\item Eigenständige Softwareentwicklung
\end{itemize}
\vspace{0.2cm}
\textbf{3. Aktivitäten des generellen Betriebs}
\begin{itemize}
	\item Shop Verwaltung
	\item Shop \textit{(Weiter-)}Entwicklung und Wartung
	\item Eigenständige Verwaltung von Software
	\item Eigenständige Überwachung von Software
	\item Kundenservice und individuelle Beratung
\end{itemize}
\vspace{0.2cm}
\textbf{4. Aktivitäten zur Verbesserung der Supply Chain}
\begin{itemize}
	\item IDE\textit{(Entwicklungsumgebung)} entwickeln
	\item Dokumentation \& API Reference
	\item Tool Entwicklung \textit{(Für Entwickler)}
\end{itemize}

\subsubsection{Schlüsselpartner}
Die Schlüsselpartner sind die Teilnehmer der Supply Chain, von denen der Software Shop direkt abhängig ist. Eine gute Beziehung zu Ihnen ist wichtig, um den Software Shop zu erhalten.
\begin{itemize}
	\item \textbf{Software Provider}\\
	Software Provider sind die Entwickler von Softwares. Durch ihre Gedankengänge werden vielfältige, individuelle Softwares möglich. Die Produkte von Software Providern dürfen \textbf{keine Sicherheitslücken enthalten}. Um dies zu unterstützen, müssen neue Entwickler den Aufbau einer Software für Fahrzeuge schnell verstehen und selber probieren können. Software Provider sind der wichtigste Partner des Software Shops und müssen auch dementsprechend beachtet und in den Entwicklungsprozess integriert werden.\\
	
	\item \textbf{Automobilhersteller}\\
	Um mit dem Shop den größtmöglichen Erfolg zu erzielen, sollte dieser auf möglichst vielen Fahrzeugen vorhanden sein. Es ist daher wichtig, möglichst viele Automobilhersteller in die Supply Chain zu integrieren und somit den Markt zu erweitern. Automobilhersteller könnten andernfalls ein Konkurrenzprodukt veröffentlichen und mit diesem den möglichen Markt einschränken.\\
	Der Software Shop sollte zudem auf jedem hergestellte Neuwagen vorinstalliert sein, um bei Verkauf des Fahrzeugs direkt verfügbar zu sein.
	
	\item \textbf{Mobilfunkbetreiber}\\
	Da ein gut ausgebautes Mobilfunknetz wichtig für die Bereitstellung von Software ist, sollten Partnerschaften mit Mobilfunkbetreibern getroffen werden, damit Bandbreite für den Download von Software günstiger sein kann aber auch damit eine notwendige Netzwerk-Infrastruktur in einzelnen Ländern errichtet werden kann.
	
\end{itemize}

\subsubsection{Kostenstruktur}
Die Pflege der Beziehungen mit den Schlüsselpartnern ist wichtig, da der Software Shop ohne sie nicht existieren könnte. So ist die Entwicklung einer IDE, von Tutorials, API Referenzen und Analysetools für Software Provider zugleich wichtig aber auch kostspielig. Auch für sonstige Events des Unternehmens \textit{(Konferenzen, Tagungen etc.)} sind hohe Kosten zu kalkulieren. \\
Es ist vorstellbar, dass Automobilhersteller prozentuale Gewinne des Shops erhalten. Diese Summe könnte anhand der Menge verkaufter Softwares auf Fahrzeugen dieser Marke berechnet werden. Weiterhin fallen Kosten für den Betrieb , die (Weiter-)Entwicklung und die Vermarktung des Shops an. Diese sind im wesentlichen für die Bezahlung von Managern, Entwicklern und Designern. Weiter fallen Stromkosten für das Betreiben von Servern an.