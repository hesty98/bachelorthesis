\section{Das Business Model Canvas und Wertschöpfungskette}\label{markt}
Damit die Verteilung von \textbf{Software} geeignet organisiert wird, bedarf es einer Plattform über welche Softwares orts- und zeitunabhängig Heruntergeladen werden können. Diese Plattform kann in Form eines Software Shops realisiert werden, über welchen Softwares von Fahrzeughaltern gekauft und installiert werden können. Der Shop stellt den Mittelsmann zwischen Fahrzeughaltern und den Softwareherstellern dar. Die installierten Softwares können unter anderem den Fahrumfang autonomer Fahrfunktionen erweitern oder das Auto mit anderen Akteuren des Straßenverkehrs verbinden. Ein bereitgestellter Shop sollte Automarkenübergreifenden entwickelt werden, da hierdurch zugleich ein großer Teil das Marktes gewonnen wird und außerdem einige Mehrwerte für Fahrzeughalter entstehen, wie die folgenden:\\

\textbf{1. Nachhaltigkeit von Fahrzeugen}\\
Nach dem verlassen des Fließbandes altert die Software eines Autos. Updates sind heutzutage nur beim Mechaniker möglich und ist zudem mit einem großen Aufwand verbunden. Durch eine Kabellose Schnittstelle sollen Fahrzeughalter gewünschte Software orts- und zeitunabhängig installieren können. Durch Softwares können Fahrzeuge möglicherweise sicherer\textit{(weniger Unfälle)} und schonender\textit{(geringere Abnutzung etc.)} gefahren werden, wodurch sich die Lebenszeit des Fahrzeugs verlängern könnte. Bleiben Fahrzeuge länger \textit{"modern"}, kann die Nachfrage an Neuwagen langfristig zurückgehen kann. Ist der Shop Automarken-übergreifend, können sämtliche Software des Shops für jeden Kunden weltweit verfügbar sein. Hierzu müssen sämtliche Fahrzeuge die Softwares installieren und kompilieren können.\\


\textbf{2. Selbstständige Erweiterung des Fahrzeugs}\\
Die schlechte Alternative zu einer kabellosen Bereitstellung von Software ist die stetige Fahrt zum Mechaniker um neue Softwares zu installieren. Hier erhält ein Fahrzeug eine festgelegte Sammlung an Softwares, der Fahrzeughalter hat also keine Auswahlmöglichkeiten. Hierdurch kann auf Dauer viel Software auf dem Fahrzeug installiert sein, die der Fahrzeughalter nicht benötigt. Durch einen Softwareshop wird es möglich, dass ein Fahrzeughalter nur die tatsächlich benötigte Software auf seinem Fahrzeug installiert kann. Des weiteren können die Kosten für Fahrzeughalter hierdurch skalierbar gehalten werden, da ein Fahrzeughalter selber entscheiden kann welche Softwares er kauft.\\

Um einen Überblick des Marktes zu geben, wird zunächst ein Business Model Canvas \textit{(BMS)} erarbeitet. Anschließend werden relevante Bausteine der Wertschöpfungskette anhand der Erkenntnisse aus dem Forschungsseminar sowie des Business Models identifiziert und deren Aufgaben erläutert. Abschließend erfolgt eine Zusammenfassung des Kapitels und es werden die technischen Konzepte vorgestellt.