\section{Das Business Model Canvas und die Wertschöpfungskette}\label{markt}
Damit die Verteilung von \textbf{Software} geeignet organisiert wird, bedarf es einer Plattform über welche diese orts- und zeitunabhängig Heruntergeladen werden können. Diese Plattform kann in Form eines Software Shops realisiert werden, über welchen Softwares von Fahrzeughaltern gekauft und installiert werden können. Ein passender Vergleichaus der Realität ist der Google Play Store. Der Shop stellt den Mittelsmann zwischen Fahrzeughaltern und den Softwareherstellern dar. Die installierten Softwares können unter anderem den Fahrumfang autonomer Fahrfunktionen erweitern oder das Auto mit anderen Akteuren des Straßenverkehrs verbinden. Ein bereitgestellter Shop sollte Automarkenübergreifenden entwickelt werden, da hierdurch zugleich ein großer Teil das Marktes gewonnen wird sowie einige Mehrwerte für Fahrzeughalter geschaffen werden wie zum Beispiel\\

\textbf{1. Nachhaltigkeit von Fahrzeugen}\\
Nach dem verlassen des Fließbandes altert die Software eines Autos. Updates sind heutzutage nur beim Mechaniker möglich und sind daher mit einem großen Zeitaufwand für Fahrzeughalter verbunden. Mittels einer Kabellose Schnittstelle können Fahrzeughalter gewünschte Software orts- und zeitunabhängig installieren. Bleibt die Software von Fahrzeuge länger \textit{\glqq aktuell\grqq}, kann die Nachfrage an Neuwagen langfristig zurückgehen kann und sich der Automobilmarkt so grundlegend ändern. Ist der Shop Automarken-übergreifend, können sämtliche Softwares des Shops für jeden Fahrzeughalter weltweit verfügbar sein.\\


\textbf{2. Selbstständige Erweiterung des Fahrzeugs}\\
Die schlechte Alternative zu einer kabellosen Bereitstellung von Software ist die stetige Fahrt zum Mechaniker. Hier erhält ein Fahrzeug eine festgelegte Sammlung an Softwares, der Fahrzeughalter hat also keine Auswahlmöglichkeiten.\footnote{quelle} Auf Dauer kann daher viel Software auf dem Fahrzeug installiert sein, die der Fahrzeughalter nicht benötigt. Durch einen Softwareshop wird es möglich, dass ein Fahrzeughalter nur die tatsächlich benötigte Software auf seinem Fahrzeug installiert hat. Des weiteren können die Kosten für Fahrzeughalter hierdurch skalierbar gehalten werden, da diese selber entscheiden welche Softwares gekauft werden.\\

Um einen Überblick des Marktes zu geben, wird zunächst ein Business Model Canvas \textit{(BMS)} erarbeitet. Anschließend werden relevante Bausteine der Wertschöpfungskette anhand der Erkenntnisse aus dem Forschungsseminar sowie des Business Models identifiziert und deren Aufgaben erläutert.