\section{Der Markt neuer Fahrumfänge}\label{markt}
Damit die Verteilung von \textbf{Software} vernünftig organisiert wird, bedarf es einer Plattform über welche Softwares angeschafft und heruntergeladen werden können. Diese zentrale Verwaltung von Softwares wird in Form eines Softwareshops realisiert. Über diesen können Softwares von Fahrzeughaltern orts-und zeitunabhängig gekauft und installiert werden. Der Shop stellt den Mittelsmann zwischen Fahrzeughaltern und den Softwareherstellern dar. Die installierten Softwares können unter anderem den Fahrumfang autonomer Fahrfunktionen erweitern und das Auto mit anderen Akteuren des Straßenverkehrs verbinden. Der Fahrzeughalter kann mit diesen Interagieren und von Ihnen bereitgestellte \textbf{Services} \textit{(z.B. Kauf eines Parktickets)} nutzen.\\
Die Bereitstellung eines Automarkenübergreifenden Softwareshops ist aus mehreren Gründen sinnvoll:\\

\textbf{1. Nachhaltigkeit von Fahrzeugen}\\
Nach dem verlassen des Fließbandes altert die Software eines Autos. Updates sind heutzutage nur beim Mechaniker möglich und ist zudem mit einem großen Aufwand verbunden. Durch eine Kabellose Schnittstelle sollen Fahrzeughalter gewünschte Software orts- und zeitunabhängig installieren können. Durch Softwares können Fahrzeuge sicherer\textit{(weniger Unfälle)} und schonender\textit{(geringere Getriebeabnutzung etc.)} gefahren werden, wodurch sich die Lebenszeit des Fahrzeugs um eine unbestimmte Zeitspanne verlängern und langfristig die Nachfrage an Neuwagen zurückgehen kann.\\


\textbf{2. Ressourcen von Fahrzeugen schonen}\\
Durch die stetig steigende Anzahl an installierten Softwares eines Fahrzeugs, werden Festplatten immer voller, die benötigte Rechenzeit kann steigen und der Fahrzeughalter hat keinen Überblick mehr über die installierten Softwares. Durch einen Softwareshop wird es möglich, dass ein Fahrzeug nur tatsächlich benötigte Software installiert, wodurch die Ressourcen geschont werden.\\

\textbf{3. Kosten für Kunden Skalierbar halten}\\
Müsste ein Fahrzeughalter jede neu entwickelte Software auf seinem Fahrzeug installieren und zusätzlich noch für diese Zahlen, können für diesen unnötige Kosten entstehen. So brauch ein Fahrzeug das nur in der Stadt fährt beispielsweise keine Software die das Off-Road fahren unterstützt. Durch die Möglichkeit einzelne Softwares orts- und zeitunabhängig zum Fahrzeug hinzufügen zu können, wird dieses Problem eliminiert.\\

\textbf{4. Verbesserung der Wettbewerbssituation}\\
Durch die Bereitstellung eines Automarkenübergreifenden Softwareshops kann ein großer Teil des Marktes gewonnen werden. Dies kann die Position des Unternehmens in diesem manifestieren und so zu neuen Einnahmequellen führen.\\\\

Um einen Überblick des Marktes zu geben, wird zunächst ein Business Model Canvas \textit{(BMS)} erarbeitet. Anschließend werden relevante Bausteine der Wertschöpfungskette anhand der Erkenntnisse aus dem Forschungsseminar sowie des Business Models identifiziert und deren Aufgaben erläutert. Abschließend erfolgt ein Überblick des Kapitels, woraufhin die technischen Konzepte vorgestellt werden.