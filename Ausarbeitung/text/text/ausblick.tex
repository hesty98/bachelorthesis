\section{Reflexion der Ergebnisse}
Die zu Beginn der Bachelorarbeit aufgestellte Forschungsfrage stellte die Anforderung, wichtige Bausteine einer Wertschöpfungskette für die Erkennung und Bereitstellung neuer Fahrfunktionen zu identifizieren und darzustellen, wie diese von Automobilherstellern umgesetzt werden können. Das erstellte Business Model Canvas wurde mit der Intention erstellt, als "Einführung" in den Markt für Fahrfunktionssoftware zu dienen. Die hierdurch zusätzlich entstandenen Inhalte haben eine sehr gute Grundlage für einen "Blick über den Tellerrand" geboten, welcher eine mögliche Zukunft von Fahrzeugen veranschaulicht.\\\\
Die in der Wertschöpfungskette identifizierten Bausteine haben viele Aufgaben eines Software Shops aufgedeckt. Einige der Bausteine wurden in dieser Arbeit aus Platzgründen nicht im Detail erläutert, damit die im Fokus stehenden Bausteine der Bedarfserkennung und Bereitstellung ausreichend detailliert werden konnten. In den Kapiteln \ref{forschungsseminar} und \ref{technische_konzepte} wurden technischen Konzepte erstellt um diese Bausteine tiefer gehend zu betrachten. Die Ergebnisse dessen sind zufriedenstellend, da sie eine guten Überblick über die verschiedenen Aufgabenfelder der Erkennung und Bereitstellung von Software bieten. Dennoch hätten einige dieser Bausteine qualitativ besser erörtert werden können, zum Beispiel durch die Integration von mehr Bausteinen in den Prototypen. Dieser hat letzten Endes die vier folgenden Bausteine der Wertschöpfungskette visualisiert:
\begin{itemize}
	\item Betreiben des Shops
	\item Automatische Erkennung von Softwarebedarf
	\item Angebotsunterbreitung
	\item (Sicherer) Download über das Internet
\end{itemize}

Durch die entwickelte Architektur können diesem jedoch weitere Bausteine und neue Anwendungsfälle mit nur geringem Zeitaufwand hinzugefügt werden. Die GUI des Fahrzeugs stellt die dort geschehenden Ereignisse gut dar und die Steuerung über Knöpfe ist zielführend, jedoch nur für den Anwendungsfall der Bachelorarbeit ausgelegt. Änderungen in der Architektur wie das Auslagern des Anwendungsfalls hätten die Erweiterung zusätzlich vereinfacht. Die Integration des Uptane-Standards in den Prototypen und die Einbindung von OpenScenario-Dateien in technische Konzepte hat die Wichtigkeit von OpenSource-Lösungen verdeutlicht für Fahrzeuge aufgezeigt.\\

Einige Ergebnisse wie zum Beispiel die Bausteine des Supply Chain Managements oder des generellen Betriebs haben keinen direkten Bezug auf die Forschungsfrage der Bachelorarbeit. Diese Problematik entstand durch eine Unterschätzung des Themas zu Beginn. Es wurde erst im Laufe der Erarbeitung deutlich, dass zum Verständnis einiger Bausteine der Bedarfserkennung und Bereitstellung weiteres Grundwissen notwendig ist, welches durch das Business Model Canvas vermittelt wurde. Die Integration dieser zusätzlichen Informationen hat letzten Endes zu einem besseren Gesamtergebnis beigetragen und viele Blickwinkel eröffnet wodurch mehrere Themen für Abschluss- und Forschungsarbeiten durch diese Arbeit vorgeschlagen werden können.\\

\section{Ausblick}
In direkter Anknüpfung an die Arbeit sollten die identifizierten Bausteine weiter erforscht und im Prototypen veranschaulicht werden. Es sollte eine GUI für den Server erstellt werden, damit auch die dortigen Entscheidungen nachvollzogen werden können. Außerdem kann die Simulation in Carla durch weitere Verkehrsteilnehmer realistischer gestaltet werden. Durch die Implementierung des "Software User Pattern Recognizer" (SUPR, Kapitel \ref{forschungsseminar}) des Fahrzeugs und des Servers würde den Prototypen vervollständigen. Ein Prototyp, der diese Anforderungen erfüllt und zudem mehrere Anwendungsfälle umfasst, kann sehr gut für die Vorstellung auf Messen oder anderweitigen Präsentationen genutzt werden, um "die Zukunft des Autofahrens" zu zeigen. \\

In der Arbeit wurde davon ausgegangen, dass Software auf einem einzelnen Fahrzeug installiert wird. Es ist vorstellbar, dass gekaufte Softwares nicht nur auf einem einzigen Fahrzeug genutzt werden sollen wozu vor allem der wachsende Car-Sharing Markt beiträgt. Damit Fahrer sich einfach auf einem einloggen können und im folgenden alle jemals von ihm gekauften Softwares verfügbar sind ist ein technisches Konzept zu erstellen. Dieses soll darstellen, wie Software nicht Fahrzeug-gebunden sondern Fahrzeughalter-gebunden gekauft und genutzt werden kann.\\
Damit Softwares auf Fahrzeugen verschiedener, möglicherweise konkurrierender Automarken  installiert und genutzt werden können muss eine Software-API konzipiert werden. Hierbei besteht die Schwierigkeit, dass selbst Fahrzeuge der selben Automarke unterschiedliche Software- und Hardware-Architekturen aufweisen und eine generalisierte Ansteuerung dieser Komponenten dementsprechend umfangreich gestalten kann.