\section{Technische Konzepte}\label{technische_konzepte}
Sollte vermutlich das Kapitel "Entwicklung des Prototypen" erstezen...?\\

\subsection{Klassifizierung von Software}\label{sw_klassifizierung}
fahrzeugshconungsgrad\\
sicherheitsgrad\\
umfangsgrad\\
durchschnittlich gewonnen zeit je Stunde (Regionsabhöngig)\\
durchschnittliche Anzahl an Interaktionen mit Umwelt pro Woche\\

\begin{itemize}
	\item[1.] Fahrfunktionsapplikationen\\
	Diese Art von Applikationen erweitern die Funktionen des Fahrzeugs im Hinblick auf von Fahrer nicht wahrnehmbare Entscheidungen. Dies können Applikationen sein wie \textit{"Fahren in der Innenstadt"}, oder \textit{"Überholen auf der Autobahn"}. Das Fahrzeug würde diese Applikationen nutzen ohne auf Nutzereingaben warten zu müssen.
	
	\item[2.] Serviceapplikationen\\
	Durch Serviceapplikationen ist ein Fahrzeug interaktiver in seiner Umwelt. Sie steigern die User-Experience (UX) der Fahrzeuginsassen, indem sie mit anderen Akteuren des Straßenverkehrs \textit{(Ampeln, Parkplatzautomaten, andere Fahrzeuge uvm.)} kommunizieren und so Mehrwerte schaffen. Serviceapplikationen können auf Nutzereingaben warten, müssen dies aber nicht zwangsläufig tun.
\end{itemize}

\subsection{Kommunikationsprotokoll}