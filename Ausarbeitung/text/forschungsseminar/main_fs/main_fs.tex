%\documentclass{IEEEtran}
\documentclass{article}
\renewcommand{\contentsname}{Inhaltsverzeichnis}
\renewcommand{\refname}{Quellenverzeichnis}
\renewcommand{\listfigurename}{Abbildungen}

%\IEEEoverridecommandlockouts
% The preceding line is only needed to identify funding in the first footnote. If that is unneeded, please comment it out.
\usepackage{capt-of}
\usepackage{cite}
\usepackage[table]{xcolor}
\usepackage[onehalfspacing]{setspace}
\usepackage{longtable}
\usepackage{geometry}
\usepackage{amsmath,amssymb,amsfonts}
\usepackage{setspace}
\usepackage{algorithmic}
\usepackage{graphicx}
\usepackage{textcomp}
\usepackage{hyperref}
\usepackage{wrapfig}
\usepackage[utf8]{inputenc}
\usepackage{subfigure}
\usepackage{svg}
\usepackage{amsmath}
\usepackage{booktabs}
\usepackage{tabularx}
%\usepackage{subfig}
%% Rechtecke um Text %%
\usepackage{mdframed}
%% eigene packages
\usepackage{float}
\usepackage{subfiles}
\usepackage[T1]{fontenc}

\geometry{
	left=3.6cm,
	right=3.6cm,
	top=2cm,
	bottom=4cm,
	bindingoffset=5mm
}



\def\BibTeX{{\rm B\kern-.05em{\sc i\kern-.025em b}\kern-.08em
    T\kern-.1667em\lower.7ex\hbox{E}\kern-.125emX}}
\begin{document}

\begin{titlepage}
    \begin{figure}[H]
    	\centering
    	\subfigure{
    		\includegraphics[width=.4\textwidth]{../pictures/uol_logo.png}
    	}        
    \end{figure}
 
    \begin{center}
        \large{\textbf{Konzept und Ansatz einer Wertschöpfungskette für die Erkennung und Bereitstellung neuer Fahrumfänge intelligenter Fahrzeuge}}\\
        \large Forschungsseminar zur gleichnamigen Bachelorarbeit
    \end{center}
    \vfill
    \large{
        An der\\
        Carl von Ossietzky Universität Oldenburg\\
        Studiengang Wirtschaftsinformatik\\
        
	    \noindent
	    Vorgelegt von Linus Hestermeyer\\ 
        \textit{linus.hestermeyer@gmail.com}\\
        Matr.Nr.: 4087097
        \\\\\\\\\\\\
        Erstprüfer: Prof. Dr. Frank Köster\\
        Zweitprüfer: Dipl.-Inform. Gerald Sauter\\
        \vfill
        \noindent
        Oldenburg, den \today
    }
\end{titlepage}
\thispagestyle{empty}
\tableofcontents
\clearpage
\pagenumbering{arabic}

\subfile{../text/forschungsseminar/einleitung_neu.tex}
\subfile{../text/forschungsseminar/bedarfserkennung.tex}
\subfile{../text/forschungsseminar/bereitstellung.tex}
\subfile{../text/forschungsseminar/technologie_scouting_und_methodik.tex}
\subfile{../text/forschungsseminar/konzept_praxis.tex}

\listoffigures
\bibliographystyle{plain}
\bibliography{library}
\subfile{../text/forschungsseminar/eides_statt.tex}
\end{document}
