\subsection{Einführung}
"Durch den Einsatz von Elektrik und Elektronik ist das Automobil in den vergangenen
Jahrzehnten stark geprägt worden, und die „Intelligenz“ in den Subsystemen hat exponentiell zugenommen." (Vgl. \cite[S. 1]{uniStuttgart}) Mit zunehmend mehr intelligenten Fahrassistenten in modernen Fahrzeugen ist es Absehbar, dass bereits in naher Zukunft Fahrzeuge die Stufe 3 bzw. teilweise Stufe 4 des Autonomen Fahrens erreichen. Ein Fahrzeug der Stufe 3 ist dazu in der Lage dazu, die Längs- und Querlenkung in bestimmten Anwendungsfällen selber übernehmen und diese so sicher zu durchfahren. Hierbei wäre allerdings noch ein Insasse notwendig, der in einem Notfall das Steuer übernehmen kann. In Stufe 4 Fahrzeugen kann das Fahrzeug die komplette Fahraufgabe in bestimmten Anwendungsfällen übernehmen.\cite[S.\, 14]{vda}\\

Erst Stufe 5 Fahrzeuge werden "vollumfänglich auf allen Straßentypen, in allen Geschwindigkeitsbereichen und unter allen Umfeldbedingungen die Fahraufgabe vollständig allein durchführen. Wann dieser Automatisierungsgrad erreicht sein wird, kann heute noch nicht benannt werden." (Vgl.\cite[S. 14]{vda}). Aufgrund dessen ist es wichtig, dass Stufe 3 und Stufe 4 Fahrzeuge in Zukunft mehr Anwendungsfälle abdecken können. Hierzu sollen diese über eine kabellose Schnittstelle orts- und zeitunabhängig Softwarepakete herunterladen können, welche das Spektrum der autonomem Fahrfunktionen des Fahrzeugs zweckgebunden erweitert. Angenommen Sie planen mit ihrem neuen Fahrzeug eine Autofahrt nach England. Spätestens ab dem Ende des Eurotunnels wäre es für das Fahrzeug nicht mehr möglich selbstständig zu Fahren, da dort Linksverkehr herrscht. Das Auto soll automatisch erkennen können, dass es ab einem bestimmten Punkt nicht mehr selbstständig fahren kann. In Folge dessen soll es eine Software zum Kauf/Miete anbieten, welche es dem Auto ermöglicht, am Linksverkehr teilnehmen zu können. Die Halter können ihr Fahrzeug dementsprechend zunehmend autonom fahren lassen, was den Marktwert des Autos nach dem Kauf steigern kann. Die Fahrzeughalter sollen bei Kauf von Software vom System unterstützt werden in Form von Vorschlägen für neue Software, die den Bedarf des Fahrers abdeckt. Diese Vorschläge sollen zu Zeitpunkten erfolgen, in denen der Verkauf einer bestimmten Software möglichst wahrscheinlich ist. Mittels dessen wird zudem unterbunden, dass der Nutzer von zu vielen Benachrichtigungen überfordert wird.\\

Damit Anwendungsfälle rasch abgedeckt werden können, bedarf es einem großen Spektrum an Software, die eben diese abdeckt. Um dies schnellstmöglich bewerkstelligen zu können ist es erforderlich, dass die Entwicklung von Softwarepakete durch Zulieferer geschehen, welche sich explizit auf diesen Markt fokussieren. Der hierdurch entstehende Seitenmarkt für die Entwicklung von Fahrtfunktionssoftware kann somit schnell wachsen und sich als Teil in der Automobilindustrie etablieren. Durch autonome Fahrfunktionen werden Autos in der Regel schonender gefahren als vom Menschen, weshalb die Lebenszeit von Autos voraussichtlich verlängert wird.\cite[(S. 16)]{vda} Ericssons Juergen Daunis sagt diesbezüglich, dass die meisten Analytiker und Führungskräfte der Meinung sind, dass der Umsatz dieser neu entstehenden Seitenmärkte weiter steigen wird und dass das traditionelle Geschäftsmodell an dem wirtschaftlichen Maximum seiner Existenz ist.\cite{bmd} \\

Die in dieser Arbeit erfassten Gliederungspunkte dienen als Grundlage für die gleichnamige Bachelorarbeit und sind im Wesentlichen als Literaturrecherche und Grundlagenerarbeitung zu verstehen. Es werden einzelne Bausteine der in der Bachelorarbeit zu erstellenden Wertschöpfungskette erstmalig benannt und konkretisiert. In Kapitel zwei wird dargestellt, wie das Auto einen Softwarebedarf erkennt, wie es einen Server hierüber informiert und wie dieser Server die richtige Software sucht. In Kapitel Drei wird zum einen eine sichere Architektur für kabellose Aktualisierungen erarbeitet. Des weiteren wird erläutert, wie Software verkauft wird und es werden Richtlinien für die Mensch-Maschine-Schnittstelle erarbeitet, auf welcher ein Angebot letzten Endes angezeigt werden soll.
